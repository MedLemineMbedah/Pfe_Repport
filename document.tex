\documentclass[a4paper, 12pt]{report}
\usepackage[top=2.5cm,bottom=2.5cm,left=2.5cm,right=2.5cm]{geometry}
%-----------------------------------------------------------------------  01
\usepackage[pdftex]{graphicx}
\usepackage{graphics}
%\usepackage{fancyhdr} 
%----------------------------------------------------------------------   02
\usepackage[utf8]{inputenc} 
\usepackage[T1]{fontenc}
\usepackage[french]{babel}
\usepackage{arabtex}
\usepackage{utf8}
\usepackage{hyperref}
\hypersetup{
	colorlinks,
	citecolor=blue,
	filecolor=blue,
	linkcolor=blue,
	urlcolor=blue
}
\usepackage{array}
\usepackage{tabularx}
\usepackage{titlesec}

\usepackage{makecell, amssymb, amsthm, mathtools, rotating, colortbl, enumitem, nomencl, latexsym, bookmark, url, subcaption, float, multirow, algorithm, algorithmic, color}
\usepackage{arabtex, utf8}
%\setcode{utf8}
\usepackage[nomain, acronym]{glossaries}




%-----------------------------------------------------------------------   03
\linespread{1.5}

%%%%%%%%%%%%%%%%%%%%%%%%%%%%%%%%%%%%%%%%%%%%%%%%%%%%%%%% fix Subsection Error %%%%%%%%%%%%%
\setcounter{tocdepth}{3}
\setcounter{secnumdepth}{3}
\usepackage{lipsum}
%%%%%%%%%%%%%%%%%%%%%%%%%%%%%%%%%% bloc of comment %%%%%%%%%%%%%%%%5
\usepackage{verbatim}


%%%%%%%%%%%%%%%%%%%%%%%%%%%best usage for fig%%%%%%%%%%%%%%%%%
\usepackage{tikz}
\usepackage[export]{adjustbox}

\usepackage{caption}
\usepackage[skip=1ex, belowskip=2ex]{subcaption}
\newcommand{\xdownarrow}[1]{%
	{\left\downarrow\vbox to #1{}\right.\kern-\nulldelimiterspace}
}
\newcommand{\differentchapter}[1]{%
	\begingroup
	\patchcmd{\@makeschapterhead}{50}{20}{}{}%
	\patchcmd{\@makeschapterhead}{40}{70}{}{}%
	\patchcmd{\@makeschapterhead}{\bfseries}{\centering\bfseries}{}{}%
	\chapter*{#1}%
	\endgroup
}

\begin{document} 
	\begin{titlepage}
		%------------------------------------------------------------------------  04
		\begin{figure}[htbp]
			\hbox{
				\hspace*{-1.4cm}
				\includegraphics[width=100px]{./Template LaTeX/Images/ISCAE.jpg}
				\hspace*{11cm}
				\includegraphics[width=80px]{./Template LaTeX/Images/logoUniv.jpg}
			}
		\end{figure}
		%-------------------------------------------------------------------------- 05
		\vspace {-4cm}
		
		\begin{center}
			\vspace{1cm}
			
			{\bf République Islamique de Mauritanie } \vspace{-0.2cm}\\
			{\bf	Ministre de l’Enseignement Supérieur et de la } \vspace{-0.2cm}\\
			
			{\bf Recherche Scientifique }\\
			
			\vspace{2cm}
			
			%------------------------------------------------------------------------- 06
			\bf{MÉMOIRE DE STAGE DE FIN D'ÉTUDES} \\
			\vspace{1cm}
			{\bf Présenté en vue de l’obtention du } \vspace{-0.2cm}\\
			{\bf Diplôme de Master Professionnel en Informatique}\vspace{-0.2cm}\\
				{\bf	Appliquée à la Gestion (MPIAG)}\\ \vspace{0.2cm}
				{\bf Par :}\\ \vspace{0.2cm}
					{\bf Mohamed Lemine Salem M’bedah (IE18689) }\\ \vspace{0.2cm}
			\huge{\textbf{Thème}}\\ 
			\noindent\rule{\textwidth}{1mm}
			\Large{\bf{CREATION D’UN SYSTEME DE GESTION DE RELATION CLIENT (CRM) ET LA CONNAISSANCE CLIENT (KYC)}}
			\noindent\rule{\textwidth}{1mm}
		\end{center}
		\vspace{0cm}
		%-------------------------------------------------------------------------- 07
		\begin{center}
			
			{\bf Encadré par :}\\
			{Dr. Cheikh Dhib} \\
			{Mr. Mboirick Mohamed} \\
			{\bf Réalisé au sein de CADORIM }
			
			
		\end{center}
		\begin{figure}[htbp]
		\hbox{
			\hspace*{5cm}
			\includegraphics[width=150px]{./Template LaTeX/Images/cado_logo.png}
		}
		\end{figure}
		
		\begin{center}
			Année universitaire 2021-2022
		\end{center}
		%--------------------------------------------------------------------------- 08
	\end{titlepage}
		\chapter*{DEDICACES} \label{chap:1Dedicaces}
		%\addcontentsline{toc}{chapter}{DEDICACES} 
		\begin{comment}
			content...
		
		Les mots ne sauront traduire ce qui est dans le cœur, mais je vais rassembler mon vocabulaire et dire : je dédie cet humble travail à mes chers parents pour leur soutien tout au long de mon parcours universitaire, qui ont été les encouragements dans les moments de détresse, l'exhortation en période de prospérité, le soutien lors de la chute et le guide lors de la montée,qui n’a pas hésité un instant à m’aider jusqu’à ce que je suis arrivé là où
		je suis maintenant. Tous les remerciements et gratitudes ne remplissent pas leur droit.
	
		
		
	\textbf{Je dédie ce modeste travail }\newline
	\textbf{A mes parents en témoignage de mon affection et de ma profonde gratitude pour leur
		soutien moral et financier et leurs encouragements ;}\newline
	\textbf{	A mes frères et sœurs ;}\newline
	\textbf{	A toute ma famille, qu’elle soit proche ou lointaine ;}\newline
	\textbf{	A tous mes amis et camarades de classes pour leur aide et conseils ;}\newline
	\textbf{	A mon tuteur pour son accueil chaleureux tout au long de mon cursus universitaire;}\newline
	\textbf{	A mon oncle depuis les Etats-Unis pour son soutien et ses encouragements durant tout
		mon cursus universitaire}
	

	\begin{center}
		\it \Large
			\raggedright Je dédie ce modeste travail \newline
			\vspace{4mm}
			\textbf{A mes parents en témoignage de mon affection et \vspace{4mm} de ma profonde gratitude pour leur soutien moral et financier et leurs encouragements ;}\newline
		\textbf{	A mes frères et sœurs ;}\newline
		\textbf{	A toute ma famille, qu’elle soit proche ou lointaine ;}\newline
		\textbf{	A tous mes amis et camarades de classes pour leur aide et conseils ;}\newline
		\textbf{	A mon tuteur pour son accueil chaleureux tout au long de mon cursus universitaire;}\newline
		\textbf{	A mon oncle depuis les Etats-Unis pour son soutien et ses encouragements durant tout
			mon cursus universitaire}
		
	
	\end{center}
\chapter*{\Huge Dédicaces}
\end{comment}
\begingroup
\large \raggedright Je dédie ce travail à :

\vspace{4mm}
	\textbf{	Mes frères et sœurs ;}
	
\vspace{4mm}
	\textbf{	Toute ma famille, qu’elle soit proche ou lointaine ;}
	
\vspace{4mm}
	\textbf{	Tous mes amis et camarades de classes pour leur aide et conseils ;}
	
\vspace{4mm}

	\textbf{	Mon tuteur pour son accueil chaleureux tout au long de mon \hspace*{0.5cm} cursus universitaire;
		}
	
\vspace{4mm}
\textbf{	Mon oncle depuis les Etats-Unis pour son soutien et ses  \newline \hspace*{0.5cm} encouragements durant tout
	mon cursus universitaire;}	
\endgroup

\vspace{8mm}
\begin{flushright}
%	\LARGE \@author
\end{flushright}
		%sa signification
			%\textbf{<< Dis, mon Seigneur, aie pitié d'eux comme ils m'ont élevé quand j'étais jeune >>}
		\thispagestyle{empty} 
		\chapter*{REMERCIEMENTS} \label{chap:1REMERCIEMENTS}
		%\addcontentsline{toc}{chapter}{REMERCIEMENTS} 
		\textit{
			Ce n’est pas parce que la tradition l’exige que cette page se trouve dans ce rapport;
			mais par ce que les gens à qui s’adressent mes remerciements les méritent vraiment.
			\newline
			J’adresse mes remerciements les plus chaleureux envers : 
			\begin{enumerate}
				 \item[•] \textbf{Dr. Cheikh Dhib,} l'ancien coordinateur de MPIAG, ainsi que \textbf{Dr. Emani Mohamed Sidi} le nouveau coordinateur et l’ensemble du corps
				 professoral et administratif de l’ISCAE.
				 Et pour avoir suivi et dirigé l’évolution de mon travail, ainsi que les encouragements, conseils et orientations réguliers qu’il m’a prodigués.
				 Je vous suis infiniment reconnaissant pour votre grande disponibilité et l’intérêt que vous avez porté à ce travail;
				  \item[•] \textbf{Mr. Mboirick Mohamed, } responsable technique et mon maître de stage, pour ça
				  disponibilité, sa gentillesse, son attention particulière à l’égard de ce travail. Et pour
				  avoir mis les moyens nécessaires au bon déroulement de ce stage de six mois et à
				  la réalisation de ce travail ; Je vous suis infiniment reconnaissant pour votre grande
				  disponibilité et l’intérêt que vous avez porté à ce travail ;
			\end{enumerate}
			Toutes les personnes qui de près ou de loin, ont contribué à la réalisation de ce travail.
		 }
	 \thispagestyle{empty}
	 \chapter*{AVANT-PROPOS} \label{chap:1AVANT-PROPOS}
	%\addcontentsline{toc}{chapter}{AVANT-PROPOS}
	L’Institut Supérieur de Comptabilité et d’Administration des Entreprises est un établissement public d’enseignement supérieur et de recherche, placé sous la tutelle du Ministère en charge de l’Enseignement Supérieur, crée en 2009, par décret N° 2009-161 du 29 avril 2009.
	\newline
	L’ISCAE compte deux (2) départements et dans le cadre de leur formation, les étudiants qui sont en fin de cycle sont tenus d’effectuer un stage pratique au sein d’une entreprise ou d’un service informatique.
	\thispagestyle{empty}
	\chapter*{RESUME} \label{chap:1Resumé}
	\begin{comment}
		content...
	Dans le cadre de l’obtention de notre diplôme de MASTER en Informatique Appliquée à la Gestion (MPIAG)
	à Institut Supérieur de comptabilité et d’administration des entreprises nous avons été
	appelés à réaliser un projet de fin d’études afin de clôturer notre formation du second cycle
	universitaire. C’est ainsi qu’on a eu l’occasion d’enrichir davantage nos connaissances
	théoriques par la création d’un système de gestion de relation client (CRM) et de kyc
	 (KNOW YOUR CUSTOMER). Ce sujet a été proposé par la start-up Cadorim qui exerce
	 Dans le domaine du transfert d’argent de l’Europe vers la Mauritanie. Pour atteindre cet
	 Objectifs, nous avons utilisé les infrastructures technologiques suivantes : *UML : pour la
	 modélisation des données *Framework Flutter : pour le front-end (Interface utilisateur).
	 *Framework Laravel : pour le back-end (Interface Administrateur).
	Le système de gestion de base de données (SGBD) est MySQL.
	Il faut noter que pour modéliser et gérer les différents diagrammes UML l'outil
	StarUML nous a été utile.
	///////////////////////////////////////////
		Le développement de ce deux systèmes s’est basé sur le
	Processus Unifié (UP) comme méthode de modélisation et UML (Unified Modelling Language)
	comme langage de modélisation.	
		
		Le but de ce travail est la reimagination de l'application CADORIM pour être à jour à l'évolution  technologie dans le monde. La start-up Cadorim qui exerce dans le domaine du transfert d’argent de l’Europe vers la Mauritanie a proposé de créer un system d'obtention  des informations de client  pour  la bonne connaissance au client (KYC) et de créer un system de gestion de relation client (CRM). Ce qui nécessite la reconstruction de l'application CADORIM, Pour faciliter l'adaptation dans l'environnement de travail  dans la société et  assurer la bonne intégration de nouveaux changements.
		\\La nouvelle version de l'application permettra la validation automatique des comptes clients d'une manière automatique.
	\end{comment}
		
			
	Le présent document est le fruit de notre travail dans le cadre du stage de fin d'études, pour lobtention du diplôme de master Professionnel en informatique 	Appliquée à la gestion (MPIAG), qui a été realise au sein de la société CADORIM . Notre travail Consiste à étudier les Framework Laravel et Flutter, afin de realise deux applications mobile.L'un est une nouveaux verssion du projet mobile CADORIM permettant la gestion de transfert d'argent  ,et  identifier électroniquement des documents comme des pièces d’identité (carte d’identité, passeportet et carte de séjour),et une fenêtre de discussion.
	Et l'autre application ce pour la gestion de service client.\newline\newline
	Afin de suivre l'évolution des techniques de l'information et l'émergence du monde mobile, noter  applications est conçue pour les différentes plateformes mobiles à savoir:Android, IOS.\newline\newline
	Nous avons commencé par l'élaboration d'une étude fonctionnelle qu'on a fini par l'élaboration d'une conception du projet.\newline\newline
	les principales fonctionnalités de CADORIM sont l'inscription des membres, le transfert d'argent, la communication avec le service client.

	
	\thispagestyle{empty}
	\chapter*{Abstract} \label{chap:1Resumé1}
This document is the result of our work in the end study's  internship,
in order to obtain engineering degree in computer science applied to management,which has been done within company CADORIM.Our mission is to study two mobile development framework(Flutter,Laravel) and realize two mobile project, one is a new version of  CADORIM application and other is for menagment of clients relationship.\newline\newline
To follow the evolution of information technology and the emergence of the mobile world,our application is designed for different mobile platforms  namely Android,IOS.\newline\newline




	
	\begin{comment}
	\thispagestyle{empty}
	\chapter*{\setcode{utf8} \<ملخص>} \label{chap:iuhds}
		\setcode{utf8}
	%To start typing in Arabic use the command arabtext
	
	\<  >%
	

	
	
		content...

	%Note that your layout must support arabic text when compiling
	\setcode{utf8}
	%To start typing in Arabic use the command arabtext
	
	\<السَلامُ عَليكم ورَحمةُ الله وبَركاته  >

	\end{comment}
	
	\thispagestyle{empty}
	\chapter*{LISTE DES ABRÉVIATIONS} \label{chap:1SIGLES ET ABREVIATIONS}
	%\addcontentsline{toc}{chapter}{SIGLES ET ABREVIATIONS}
	Je présente ici certains sigles et abréviations que nous utiliserons dans le document.
	\newline\newline
	\textbf{ISCAE} : Institut Supérieur de Comptabilité et d’Administration des Entreprises \newline
	\textbf{MPIAG} : Master Professionnel en Informatique Appliquée à la Gestion \newline
	\textbf{API} : Application Programming Interface \newline
	\textbf{MVC} : Modèle Vue Contrôleur \newline
	\textbf{HTTP} : Hypertext Transfer Protocol \newline
	\textbf{HTTPS} : HyperText Transfer Protocol Secure \newline
	\textbf{UML} : Unified Modeling Language \newline
	\textbf{SQL} : Structured Query Language \newline
	\textbf{JSON} : JavaScript Object Notation \newline
	\textbf{XP} : eXtreme Programming \newline
	\textbf{IIS} : Internet Information Services \newline
	\textbf{DAO} : Data Access Object \newline
	\textbf{AAB} : Android App Bundles \newline
	\textbf{APK} : Android Package Kit \newline
	\textbf{IDE} : Integrated Development Environment (Environnement de Développement Intégré) \newline
	\textbf{MERISE} : Méthode d’Étude et de Réalisation Informatique par les Sous-Ensembles ou pour les Systèmes d’Entreprise \newline
%	\textbf{OPT} : One-Time Password \newline
	\textbf{SGBD} : Système de Gestion de Bases de Données \newline
	\textbf{SGBDR} : Système de Gestion de Bases de Données Relationnelle \newline
%	\textbf{SOA} : Service Oriented Architecture \newline
	\textbf{OCR} : Optical Character Recognition  \newline
	\textbf{MRZ} : Machine-Readable Zone   \newline
	\textbf{UX} : User Experience
	 \newline  
	\textbf{UI} : User Interface  
	\thispagestyle{empty}
	
	\tableofcontents
	\listoffigures
	\thispagestyle{empty}
	
\chapter*{Introduction générale}
\addcontentsline{toc}{chapter}{Introduction générale}
Actuellement, à travers les progrès de la technologie, le smartphone est devenu un outil
indispensable de travail qui peut apporter un plus dans la vie professionnelle et sociale de tout
un chacun. D’ailleurs cela s’illustre parfaitement dans le domaine du Mobile Money qui est
l’une des plus grandes évolutions dans le secteur consacré aux transfert d'argent 
de l’europe vers l'afrique où une grande partie de la population de 
l'afrique  n'a pas de compte bancaire.\newline
Au fur et à mesure on entend désormais partout, le processus KYC (Know Your Customer) se déploie dans de nombreux secteurs d’activité. Véritable atout pour combattre la fraude et l’usurpation d’identité, un processus KYC, lorsqu’il est correctement déployé, offre aujourd’hui une multitude d’avantages aux utilisateurs comme aux entreprises.\newline
Dans notre société, l’information est devenue un élément à la fois stratégique pour développer les
activités, et essentiel pour assurer un avantage concurrentiel (optimisation des coûts, meilleure
satisfaction client…) aux entités qui savent l’utiliser. C’est ce constat qui explique pourquoi les
entreprises cherchent aujourd’hui à mettre en place des systèmes de collecte et de traitement de
données toujours plus performants.
De même, la satisfaction du client est plus que jamais au centre des préoccupations des entreprises et
se concrétise par une gestion personnalisée de la relation client : comprendre les clients et leurs
attentes, les fdéliser, les inciter à consommer davantage. Le CRM, Customer Relationship
Management (GRC en français) a pour objet d'identifer, attirer et conserver les meilleurs clients et
d'en retirer chiffre d'affaire et rentabilité.
Ainsi le CRM englobe l'ensemble des activités et des processus que doit mettre en place une entreprise
pour interagir avec ses clients et ses prospects afn de leur fournir des produits et des services adéquats
au bon moment. Les entreprises ont de plus en plus recours à une approche de type de CRM, afn de se 
différencier. En effet, la banalisation de l'offre, une exigence accrue du client conduisent les
entreprises à faire évoluer leur offre dans le sens d'une plus grande personnalisation. Afn de parvenir
à cet objectif, l'entreprise est tenue de s'adapter à la profusion des canaux d'accès parallèles et en
particulier Internet.
L'arrivée des Nouvelles Technologies de l'Information et de la Communication a en effet un impact
très important sur les attitudes et les stratégies des entreprises face au CRM. Si bien que l'on peut se
demander si l'E-CRM, la gestion de la relation client par Internet constitue une véritable révolution
pour le CRM.





\begin{comment}
On l’entend désormais partout, le processus KYC (Know Your Customer) se déploie dans de nombreux secteurs d’activité. Véritable atout pour combattre la fraude et l’usurpation d’identité, un processus KYC, lorsqu’il est correctement déployé, offre aujourd’hui une multitude d’avantages aux utilisateurs comme aux entreprises. Mais quelles fonctions et quels usages se cachent derrière cet acronyme ? Comment a-t-il vu le jour et comment évolue-t-il face à la digitalisation des entreprises ? Découvrez, à travers ce guide, les atouts qu’un processus KYC automatisé peut apporter, ainsi que les différentes étapes qui en font une réussite.
\end{comment}
	%\let\cleardoublepage\clearpage

\chapter{Introduction générale}
\label{sec:DescriptionDuProjet}

Savoir qui est votre client et adopter des protocoles pour prévenir la criminalité financière sont des défis permanents pour les institutions financières. De manière significative, les institutions financières (y compris les banques, les coopératives de crédit et les sociétés financières du Fortune 50) doivent se conformer à un ensemble des réglementations de plus en plus complexes pour la vérification de l'identité des clients appelée KYC.

KYC, également connu sous le nom de "Know Your Customer" ou "Know Your Client", est un ensemble de procédures permettant de vérifier l'identité d'un client avant ou pendant les transactions avec les banques et autres institutions financières. Le respect des réglementations KYC peut aider à tenir à distance le blanchiment d'argent, le financement du terrorisme et d'autres stratagèmes de fraude courants. En vérifiant d'abord l'identité et les intentions d'un client au moment de l'ouverture du compte, puis en comprenant ses habitudes de transaction, les institutions financières sont en mesure d'identifier plus précisément les activités suspectes. 

Les institutions financières sont soumises à des normes de plus en plus strictes en matière de lois KYC. Ils doivent dépenser plus d'argent pour se conformer à KYC ou être passibles de lourdes amendes. Ces réglementations signifient que presque toutes les entreprises, plateformes ou organisations qui interagissent avec une institution financière pour ouvrir un compte ou effectuer des transactions devront se conformer à ces obligations.

La gestion de la relation client (CRM) est la combinaison de pratiques, de stratégies et de technologies que les entreprises utilisent pour gérer et analyser les interactions et les données client tout au long du cycle de vie du client. L'objectif est d'améliorer les relations de service client, de contribuer à la fidélisation de la clientèle et de stimuler la croissance des ventes. Les systèmes CRM compilent les données client à travers différents canaux, ou points de contact, entre le client et l'entreprise, qui peuvent inclure le site Web de l'entreprise, le téléphone, le chat en direct, le publipostage, les supports marketing et les réseaux sociaux. Les systèmes CRM peuvent également donner aux membres du personnel en contact avec les clients des informations détaillées sur les informations personnelles des clients, l'historique des achats, les préférences et les préoccupations d'achat.


\section{Motivations}    

KYC est un moyen de rendre la vérification de l'identité des clients plus précise et moins vulnérable à la fraude.

KYC doivent être effectuées lors de l'intégration d'un nouveau client, mais il est préférable de répéter ces vérifications de temps en temps, pour s'assurer que tout est comme il se doit. En surveillant les comptes clients de cette manière, les comportements suspects peuvent être signalés plus rapidement.

Un système CRM fournit des flux de travail automatisés qui permettent à votre équipe marketing de consacrer plus de temps à des tâches stratégiques, telles que la création de campagnes marketing qui résonnent, l'analyse des données de ces campagnes et le test de différentes approches basées sur ces analyses. Les agents du service client peuvent passer leur temps à travailler avec des clients qui ont des questions, des problèmes ou des besoins plus complexes. En bref, avec des processus de service client plus efficaces, les entreprises peuvent établir de meilleures relations avec leurs clients.

\section{Problématiques}	

En réalité, la réalisation d'une application,qui applique le principe de KYC et integre un  système CRM,
nécessite
de faire face à des problématiques diverses et complexes. Ainsi, la société a décidé de se contenter,
dans un premier temps, Mise en place d’un système d’extration des donnees à partir des images (carte d'identité ou passeport) et traitement des ces donnees.
Ce sujet soulève de nombreuses questions aux implications différentes. Comment peut extraire le texte apartir de l'image? Comment sera-t-il traité ? Comment peut-il être utilisé dans le principe KYC ? Comment pouvons-nous obtenir un système CRM intégré ?


\section{Objectifs}

La mise en place d'une application pour appliquer l'ide de KYC en basant sur les différent technologie disponible . En basan sur l'extraction du text apartir d'une imange OCR on peut extracter la code MRZ apartir d'une imange du piece d'idendite ou passport est passe le code a un algorithem qui permer de d'etecter les information personnel.



	\chapter{Présentation de la société}
\label{chap:introduction}
%\pagenumbering{arabic}
\section{Introduction}
\begin{figure}[h]
	\includegraphics[scale=0.14]{/home/mohamed/mes_fichier/CADORIM/LaTeX/pfe/Template LaTeX/Images/cado_logo.png}
	\centering
	\caption{CADOROM}
\end{figure}
CADORIM est une société de transfert d’argent mauritanienne basée à Nouakchott,
fondée fin 2018 par un entrepreneur mauritanien, titulaire d'un doctorat en
mathématiques,
CADORIM consiste a transférer de l’argent depuis n’importe quel pays dans le
monde vers ses proches en Mauritanie. Notre objectif et de fourni une plateforme
numérique permet à l’utilisateur de régler ses commandes en toute sécurité et
confidentialité assurée par le service de PayPal qui est mondialement connu pour sa
fiabilité et simplicité.t Pour effectuer un paiement il suffit d'une simple carte bancaire
ou un compte PayPal . et une éventuelle possibilité de virement bancaire.
CADORIM a été élu comme le champion de Banque Centrale de Mauritanie (BCM )
1ère édition 2019 Fintech Challenge,
Le siège social de CADORIM est situé à marche capital , Nouakchott, Mauritanie,
immatriculée au registre du commerce.
\section{Missions}
CADORIM offre une large palette de prestations organisées autour des activités suivantes :
\begin{enumerate}
	
	\item Maintenance et amélioration de leurs propres applications (CadoRim et MauriPay)
	\item Développement des applications 
	\item Des agences des reçoivent d'argent et de service client
	
\end{enumerate}
	%\chapter{Analyse et Spécification des Besoins}
\label{sec:Analyse et Spécification des Besoins}
\section{Introduction}
\section{User expérience}
\section{Différence entre UX et UI}
\section{Mise en place du processus}
\subsection{L’atelier (workshop)}
	\chapter{Environnement de développement}
\label{sec:EnvironnementDeTravail}
Durant la réalisation de ce projet, nous avons essayé d’utiliser différents
outils de développement, d’une part afin de rendre la tâche de la
réalisation plus facile, d’autre part pour que notre système soit robuste et
répond parfaitement a nos besoins , et que nos interfaces soient claires et
faciles à utiliser.
\section{Gestion du projet}
Cette section est pour la présentation de mé-stratégie de fonctionnement et le programme de gestion de projets que j'ai utilisé pour lancer, structurer et gérer le travail sur ma solution.
\subsection{Méthodologie de travail}
C'est la façon de conduire un processus de développement. Il s’agit d'un procédé un assemblage d’étapes ou procédures à mettre en œuvre dans un raisonnement méthodologique, accompagné d’outils et de techniques. L' application d'une méthode est incontournable dans l'entreprise de tout projet, spécialement dans la réalisation de projets informatiques. \newline
L'obligation de l'utilisation de ces méthodes trouve sa justification dans un certain nombre de facteurs :
\begin{enumerate}
	\item Plusieurs projets informatiques dans les passes sont échoués gras à un manque d'organisation ou la non-satisfaction des besoins;
	\item La révolte de l'industrie des applications provoquée par des défaillances logicielles qui ont introduit de nouveaux éléments pour assurer la qualité des applications : le génie logiciel ;
	\item Diverses exigences liées au coût, au temps et aux complexités des logiciels informatiques.
\end{enumerate}

L'application de procédé de développement de logiciels permet ainsi la préparation de systèmes informatiques de manière crédible et faisable tout en répondant à l'ensemble des obligations du client et du génie logiciel.
\newline \newline
Il existe diverses méthodes de développement informatique. J'illustre deux approches : l’approche traditionnelle et l’approche agile. Les deux approches se distinguent  dans la façon de diviser le projet. Les méthodes  rationnelles ou opérationnelles ou encore traditionnelles se sont imposé les premier.
\subsubsection{L'approche traditionnelle}
Cette approche s’aviser directement de l’architecture des ordinateurs. Les méthodes traditionnelles prêchent une démarche purement planifiée avec un enchaînement d'activités bien définies. La suite des activités et le plan doivent être clairement respectés et aucun changement n’est permis. Il est prévu du client une norme des besoins globaux, détaillée, claire, précise et validée en entrée. Ainsi, tout doit être prévisible, du début du projet à la livraison du produit, d’où l’appellation de méthodes prédictives. \newline
D'après le plan approuvé les méthodes cartésiennes proposent plusieurs modèles d’exécution des activités du projet :
\begin{enumerate}
	\item Le \textbf{modèle en cascade} : Sur ce modèle, la procédure de développement est découpée séquentiellement et de façon linéaire selon les activités intégrales du cycle de vie du développement logiciel : l’analyse, la conception, le codage et les tests. Le plan de déavancement des phases (planification prédictive) est établi en tout début de processus. La transition à une phase donnée n’est fait que si le résultat de la phase précédente a été validé et estimé satisfaisant par le client et les utilisateurs.
	\item Le \textbf{modèle en V} : Le cycle en V est à la base de tout développement logiciel, il en représente les activités intrinsèques. Il tient d'avantage compte de la réalité que le modèle en cascade, le processus de développement n’est pas réduit à un enchainement de tâches séquentielles. Le modèle en V permet d’anticiper sur les phases ultérieures de développement du produit en particulier les plans de test de qualifications et de performance.
\end{enumerate}
Parmi les méthodes traditionnelles, nous pouvons citer : SADT, CORIG, …
\subsubsection{L'approche agile}
Cette approche est définie par les concepts suivants : la simplicité, la légèreté, la souplesse, un lien fort avec le client. C’est dans cette optique que certains apparentent le développement agile aux notions de flexibilité, de rétroaction et d’adaptation au changement rapide et continu.
\newline
Une méthode agile est une approche itérative et incrémentale, qui est menée dans un esprit collaboratif avec juste ce qu’il faut de formalisme. Elle génère un produit de haute qualité tout en prenant en compte l’évolution des besoins des clients et en anticipant sur les risques. Il y’a continuellement des aller et retour avec le client. L’application logicielle est livrée par version incrémentale. Les versions successives sont aussi fiables que le livrable final en termes de tests et de validation. En quelque sorte le processus est déroulé comme un enchaînement de « mini-cascades ». A chaque nouvelle itération, l’ensemble de l’architecture et de la conception logicielle est reconsidéré, le code est retravaillé.
\newline
Les méthodes agiles aspirent donc à améliorer la réactivité et l’adaptabilité des sociétés de logiciels et constituent un moyen de survie dans un environnement instable en s’accompagnant des valeurs suivantes :
\begin{enumerate}
	\item Les individus et les interactions plutôt que les processus et les outils;
	\item L’application fonctionnelle plutôt que la documentation compréhensive;
	\item La collaboration avec le client plutôt que la négociation des contrats;
	\item La réponse au changement plutôt que le suivi d’un plan.
	\newline
\end{enumerate}
L’agilité comprend plusieurs courants de pensée qui ont conduits à des méthodes différentes, reposant sur les mêmes concepts mais présentant des singularités. Les méthodes Scrum, Kanban, et XP (eXtreme Programming) sont des exemples de ces méthodes.
\newline\newline

\textbf{La méthode SCRUM}
\newline
\textbf{Scrum} est une méthode agile de gestion de projet qui permet de produire la plus grande valeur métier dans la durée la plus courte. Elle a pour objectif d’améliorer la cohésion de l’équipe et la rapidité du processus de développement. Le nom Scrum renvoie à une pratique généralement connue au rugby signifiant la « mêlée ». \newline
Cette méthode qualifie un ensemble de rôles, d’instruments de gestion et de pratiques managériales favorisant un environnement basé sur la transparence, l’inspection, le suivi et l’adaptation. Le cycle de vie d’un projet Scrum peut être découpé en trois parties :
\begin{enumerate}
	\item Phase d'\textbf{initiation ou démarrage} : il s’agit d’une phase linéaire où l’on définit le périmètre fonctionnel du système et la liste des fonctionnalités (\textbf{Backlog}) agencées par ordre de priorité, d’effort, de complexité et de risque. C’est aussi à ce niveau que l’architecture est définie.
	
	\item Phase de \textbf{développement} est un processus empirique : le projet est découpé en cycles itératifs d’une durée de deux semaines ou \textbf{sprints}. Chaque sprint regroupe une ou plusieurs fonctionnalités du Backlog. Tout au long de cette phase, le travail réalisé est mesuré et contrôlé et une amélioration constante du prototype est faite.
	
	\item Phase de \textbf{Clôtures} est une phase linéaire de gestion de la livraison du produit final.
\end{enumerate}
La figure \ref{Scrum} montre l’articulation générale de Scrum. \newline
\begin{figure}[h]
	\includegraphics[width=15cm, height=5cm]{./Template LaTeX/Images/scrum1.jpeg}
	\centering
	\caption{Articulation générale de la méthode Scrum}
	\label{Scrum}
\end{figure}
Les responsabilités managériales sont réparties sur trois rôles fondamentaux :
\begin{enumerate}
	\item \textbf{Scrum Master}
	\item \textbf{Product Owner}
	\item \textbf{Équipe Scrum}
	\newline
\end{enumerate}
Les artéfacts et pratiques de Scrum
\begin{enumerate}
	\item \textbf{Product Backlog} : état courant des tâches à accomplir;
	\item \textbf{Sprint} : itération de deux semaines;
	\item \textbf{Effort-Estimation} : permanente sur les entrées du Backlog;
	\item \textbf{Sprint Backlog} : Product Backlog limité au sprint en cours;
	\item \textbf{Daily Scrum Meeting} : ce qui a été fait, ce qui reste à faire;
	\item \textbf{Sprint Review Meeting} : Présentation des résultats du Sprint. \ref{Scrum}
	\newline \newline \newline
\end{enumerate}
\textbf{SCRUM contre KANBAN}
\newline
Scrum est plus prescriptif que Kanban, qui évite de définir les rôles et les équipes et qui n’a pas de structure formelle de réunions. Kanban ne prescrit pas non plus d’itérations – bien qu’elles puissent être incorporées si vous le souhaitez. \newline
Les techniques de visualisation des processus de Kanban le rendent idéal pour les équipes colocalisées qui travaillent sur un backlog d’éléments sujets à des changements fréquents (par exemple, Kanban est souvent utilisé par les équipes de support). \newline
Le tableau Kanban est cependant souvent adopté par les équipes Scrum sous la forme d’un tableau de tâches et est utilisé pour suivre la progression tout au long d’un sprint. \newline
La limite de la règle Work In Progress dans Kanban la rend également adaptée aux équipes ayant des ressources limitées ou lorsque l’entrée de chaque membre est requise sur chaque élément. Cela pourrait s’appliquer, par exemple, à une équipe de communication au sein d’une grande organisation. \newline
Alors que Scrum limite la quantité de travail dans chaque sprint, la charge de travail est déterminée par l’estimation relative de la taille de chaque histoire (en points) et est approuvée par l’équipe Scrum à chaque session de planification. \newline
Alors qu’une équipe Kanban suit le « temps de cycle » et optimise les délais d’exécution aussi courts et prévisibles que possible, une équipe Scrum vise à améliorer son rendement sur les sprints successifs et à améliorer la « vélocité » de l’équipe (le nombre de points d’estimation relatifs complétés dans un sprint). Cela rend sans doute Scrum plus adapté à la mise à l’échelle – il semble certainement plus familier et prévisible, ce qui peut être rassurant pour les grandes organisations. \newline\newline
\textbf{SCRUM contre XP}
\newline
Dans Scrum, les équipes et les réunions sont assez gravées dans le marbre \footnote{Dans l’antiquité, les engagements pour la constructions de bâtiments importants étaient gravés sur des plaques de marbre (Athènes : arsenal du Pirée, Delphes). Les travaux s’étendant sur de nombreuses années, on ne pouvait faire confiance aux tablettes de cire ou aux papyrus. Sur ces plaques, on définissait par exemple la grandeur du bâtiment ou le montant des amendes pour les retards. Ce qui n’était pas « gravé dans le marbre » n’était donc pas contractuel. Voir le lien \href{https://fr.wiktionary.org/wiki/graver_dans_le_marbre}{https://fr.wiktionary.org/wiki/graver-dans-le-marbre}} alors que la question de savoir comment le travail est réellement fait est laissée aux équipes pour décider elles-mêmes. XP, d’autre part, est livré avec un ensemble de pratiques de base qui pourraient sembler accablantes pour le débutant Agile. \newline
On pourrait dire que Scrum est une méthodologie, qui est plus concernée par la productivité tandis que XP est plus préoccupé par l’ingénierie. \newline
La valeur que les pratiques XP peuvent ajouter est incontestable et de nombreuses organisations qui utilisent Scrum adoptent la programmation par paires, le développement piloté par les tests et le refactoring comme des pratiques qui améliorent la qualité, accélèrent le processus de publication et / ou réduisent le besoin de revoir le travail en raison de la dette technique. \newline
Outre les itérations plus courtes, d’autres éléments importants qui différencient XP de Scrum sont les suivants :
\begin{enumerate}
	\item Les équipes XP travaillent sur des éléments dans un ordre de priorité strict alors qu’une équipe Scrum ne s’attaque pas nécessairement à chaque élément dans l’ordre de priorité une fois dans le sprint;
	\item Les équipes XP peuvent intégrer de nouveaux éléments de travail dans une itération et changer d’éléments de taille équivalente (tant qu’ils n’ont pas été démarrés) si le client décide d’une nouvelle priorité.
	
	%%%%%%%%%%%%%%%%%%%%%%%%%%%%%%%%%%%%%%%%%%%%%%%%%%%%%%%%%%%%%%%%%%%%%%%%%%%%%%%%%%%%%%%%%%%%%%
\end{enumerate}
En termes de similitudes, le rôle du client dans XP est très similaire à celui du Product Owner dans Scrum – en ce sens qu’ils aident à écrire des user stories, à les hiérarchiser et sont toujours disponibles pour les développeurs – bien que moins bien définis. \newline
Scrum et XP imposent tous deux une réunion debout quotidienne. Bien que les deux soulignent l’importance de la co-localisation, seul XP le rend décisif. Voir le site \href{https://manifesto.co.uk/kanban-vs-scrum-vs-xp-an-agile-comparison/}{https://manifesto.co.uk/kanban-vs-scrum-vs-xp-an-agile-comparison/}.


\subsection{Logiciel de gestion du projet : Trello}
Trello est un outil de gestion de projet en ligne, lancé en septembre 2011 et inspiré par la méthode Kanban. Il repose sur une organisation des projets en planches listant des cartes, chacune représentant des tâches. Les cartes sont assignables à des utilisateurs et sont mobiles d'une planche à l'autre, traduisant leur avancement. Pour en savoir plus, veillez visiter le lien \href{https://fr.wikipedia.org/wiki/Trello}{https://fr.wikipedia.org/wiki/Trello}.

\section{Conception}
Dans cette section, je présente le langage et le logiciel de modélisation que j'ai utilisé pour concevoir notre solution.

\subsection{Langage de modélisation : UML}
UML est un langage de modélisation orientée objet permettant aux développeurs de modéliser un système d’information en considérant plusieurs vues chacune reflétant un aspect comportemental \footnote{Diagramme de cas d’utilisation, diagramme d’activité, diagramme de séquence, diagramme d’interaction, etc.} ou structurel \footnote{Diagramme de classe, diagramme de composants, diagramme de déploiement, diagramme de structure composite, etc.} du système.
\newline
En effet, nous avons opté pour UML au détriment de la MERISE car nous avons besoin
d’une approche de conception prenant en considération l’aspect orienté objet pour :
\begin{enumerate}
	\item Pouvoir mettre le focus sur le rôle temporel des instances d’objets lors de déclenchement desactions (à travers le diagramme de séquence) ;
	\item Faciliter par la suite la génération des classes DAO à partir du diagramme de classes.
\end{enumerate}
Certes, UML est très riche en matière de modélisation et propose au total 13 diagrammes chacun fournissant une vision particulière du système à concevoir. Dans notre contexte, je me suis limités à 4 diagrammes explicités sur le tableau \ref{3.1}.

\subsection{Logiciel de modélisation}
Modelio est un logiciel open source et multiplateforme permettant, entre autres, la modélisation UML et Business Process Model and Notation (BPMN). Pour en savoir plus, veuillez visiter le lien \href{https://www.modelio.org/about-modelio/features.html}{https://www.modelio.org/about-modelio/features.html}.
\newline
Sans doute, les logiciels de modélisation UML sont nombreux, à savoir, Visual Paradigm, Eclipse Papyrus, StarUML, PowerDesigner, Umbrello, etc. Vu que les diagrammes UML que nous voulons réaliser sont disponibles dans tous ces logiciels, il n’y avait pas en effet un choix à argumenter car
tous les choix étaient satisfaisants. Mais, de façon subjective, nous pouvons préciser que l’avantage de Modelio dans notre contexte est le fait que je m'y suis	 déjà habitués. Les diagrammes que nous avons réalisés avec Modelio sont ceux mentionnés ci-après.
\newline\newline
\begin{table}[h]
	\begin{tabular}{|m{6cm}|m{10cm}|}
		\hline
		\textbf{Diagramme} & \textbf{Rôle} \\
		\hline
		Diagramme de cas d’utilisation & Présenter les acteurs du système, ses fonctionnalités, les relations entre les acteurs et entre les fonctionnalités. \\
		\hline
		Diagramme d’activité & Déterminer l’enchaînement des différentes étapes qui composent une fonctionnalité du système. \\
		\hline
		Diagramme de séquence & Fournir une vue détaillée du diagramme d’activité en mettant le focus sur l’ordre chronologique et sur les objets crées et les méthodes appelées. \\
		\hline
		Diagramme de classe & Représenter la structure interne du système sous forme de classes et d’interfaces et préciser les différentes relations entre elles. \\
		\hline
		
	\end{tabular}
	\caption{Rôles des diagrammes UML utilisés.}
	\label{3.1}
\end{table}

\section{Implémentation}
Dans cette section, je présente les langages, les logiciels, les frameworks et les motifs d’architecture que j'ai utilisé.

\subsection{Front-end}
\subsubsection{Editeur de texte : VS Code}
VS Code (Visual Studio Code) est un éditeur de code source réalisé par Microsoft pour Windows, Linux et macOS. [9] Les fonctionnalités incluent la prise en charge du débogage, lacoloration syntaxique, la saisie semi-automatique intelligente du code, les extraits de code, la refactorisation du code et Gitintégré. Les utilisateurs peuvent modifier le thème,les raccourcis clavier,les préférences et installer des extensions qui ajoutent des fonctionnalités supplémentaires. Pour en savoir plus, veillez visiter le lien \href{https://en.wikipedia.org/wiki/Visual_Studio_Code}{https://en.wikipedia.org/wiki/VisualStudioCode}.
\subsubsection{Languages}
\begin{table}[h]
	\begin{tabular}{|m{6cm}|m{10cm}|}
		\hline
		\textbf{Langage} & \textbf{Contexte d’utilisation} \\
		\hline
		Dart & Création des applications moible\\
		\hline
		PHP  & Langage de programmation libre, principalement utilisé pour produire des pages Web dynamiques via un serveur HTTP\\
		\hline
		
	\end{tabular}
	\caption{Contexte d’utilisation des différents langages utilisés.}
\end{table}
\subsubsection{Framework : Flutter}
\label{Flutter}
Flutter est un cadre de développement d’applications mobiles open source permettant de développer des applications mobiles natives Andriod et iOS en un seul code.

Flutter a été introduit par Google. La version stable de Flutter est Flutter 1.0 qui a été publiée le 4 décembre 2018. Le ciel est la première application Flutter qui a fonctionné dans l’OS Andriod. \href{https://www.claudebueno.com/technologies/introduction-a-flutter.htm}.
\subsubsection{Framework Web :  Laravel}
\label{Laravel}
Laravel est un framework d'application Web avec une syntaxe expressive et élégante. Nous croyons que le développement doit être une expérience agréable et créative pour être vraiment épanouissante. Laravel tente de simplifier le développement en facilitant les tâches courantes utilisées dans la majorité des projets Web, telles que l'authentification, le routage, les sessions et la mise en cache.

Laravel vise à rendre le processus de développement agréable pour le développeur sans sacrifier les fonctionnalités de l'application. Les développeurs heureux font le meilleur code. À cette fin, nous avons tenté de combiner le meilleur de ce que nous avons vu dans d'autres frameworks Web, y compris des frameworks implémentés dans d'autres langages, tels que Ruby on Rails, ASP.NET MVC et Sinatra.

Laravel est accessible, mais puissant, fournissant des outils puissants nécessaires pour les applications volumineuses et robustes. Une superbe inversion du conteneur de contrôle, un système de migration expressif et une prise en charge des tests unitaires étroitement intégrée vous offrent les outils dont vous avez besoin pour créer n'importe quelle application qui vous est confiée.
\href{https://laravel.com/docs/4.2/introduction#laravel-philosophy}.
\subsubsection{IDE : Android Studio}
Android Studio est l’IDE officiel pour le système d’exploitation Android de Google,construit sur le logiciel IntelliJ IDEA de JetBrainset conçu spécifiquement pour le développement Android. Pour en savoir plus veillez visiter le lien \newline \href{https://en.wikipedia.org/wiki/Android_Studio}{https://en.wikipedia.org/wiki/AndroidStudio}. \newline
Essentiellement, je l'ai utilisé pour lancer l'application sur android.
\subsection{Back-End}
\subsubsection{IDE : Visual Studio}
Microsoft Visual Studio est un IDE de Microsoft. Il est utilisé pour développer des programmes informatiques, ainsi que des sites Web, des applications Web, des services Web et des applications mobiles. Pour en savoir plus, veillez visiter le lien \href{https://en.wikipedia.org/wiki/Microsoft_Visual_Studio}{https://en.wikipedia.org/wiki/MicrosoftVisualStudio}.
\subsubsection{MVC}
\label{3.3.3.1}
L'objectif du motif MVC (Model-View-Controller ou Modèle-Vue-Contrôleur) est un modèle dans la conception de logiciels. Il met l'accent sur la séparation entre la logique métier et l'affichage du logiciel. Cette «séparation des préoccupations» permet une meilleure répartition du travail et une maintenance améliorée. Le tableau ci-dessous en présente une explication.
\newline\newline
\begin{table}[h]
	\begin{tabular}{|m{6cm}|m{10cm}|}
		\hline
		\textbf{Couche logicielle} & \textbf{Rôle} \\
		\hline
		Modèle & Gère les données et la logique métier \\
		\hline
		Vue & Gère la disposition et l'affichage \\
		\hline
		contrôleur & Chemine les commandes des parties "model" et "view" \\
		\hline
	\end{tabular}
	\caption{Rôle des trois couches logicielles du motif MVC.}
\end{table}
\newline La figure ci-dessous explique les moments d'intervention de chaque couche du motif MVC.
\begin{figure}[h]
	\includegraphics[width=15cm, height=8cm]{./Template LaTeX/Images/mvc2.png}
	\centering
	\caption{Rôle des trois couches logicielles du motif MVC}
\end{figure}
\section{Sécurité}
Dans cette section, je présente les algorithmes et techniques de chiffrement que nous avons
utilisés pour sécuriser davantage l'application.
\subsection{Hachage des mots de passe}
	\chapter{L’implémentation}

%\section{Environnement de développement}
\section{Environnement de travail}
Durant la réalisation de ce projet, nous avons essayé d’utiliser différents
outils de développement, d’une part afin de rendre la tâche de la
réalisation plus facile, d’autre part pour que notre système soit robuste et
répond parfaitement a nos besoins , et que nos interfaces soient claires et
faciles à utiliser.
%\subsection{Environnement logiciel}

\begin{comment}
	content...

\subsection{Front-end}
C’est le développement coté client autrement dit la
partie du code reçu par le client. On rappelle que le client désigne
un navigateur web.
On a choisi Flutter comme Framework pour développer la partie
front de notre application mobile.
Flutter est un framework de développement d’applications mobiles open source de Google. La principale raison de sa popularité est qu’il prend en charge la création 		   
d’applications multiplateformes. Flutter est également utilisé pour créer des apps interactives qui s’exécutent sur des pages web ou sur le bureau.
\end{comment}
\subsection{Editeur de texte : VS Code}
\begin{comment}
	

\begin{figure}[h]
	\includegraphics[scale=0.4]{./Template LaTeX/Images/VSCode.jpg}
	\centering
	\caption{VS Code}
\end{figure}

\end{comment}
VS Code (Visual Studio Code) est un éditeur de code source réalisé par Microsoft pour Win-
dows, Linux et macOS.Les fonctionnalités incluent la prise en charge du débogage, lacoloration
syntaxique, la saisie semi-automatique intelligente du code, les extraits de code, la refactorisation du
code et Gitintégré. Les utilisateurs peuvent modifier le thème,les raccourcis clavier,les préférences et
installer des extensions qui ajoutent des fonctionnalités supplémentaires.\newline Pour en savoir plus, veillez
visiter le lien : \href{https://en.wikipedia.org/wiki/VisualStudioCode}{https://en.wikipedia.org/wiki/VisualStudioCode}
\subsection{IDE : Android Studio}
\begin{comment}

\begin{figure}[h]
	\includegraphics[scale=0.2]{./Template LaTeX/Images/Android_Studio_Icon_3.6.svg}
	\centering
	\caption{Android Studio}
\end{figure}
\end{comment}
Android Studio est un environnement de développement pour développer des applications mobiles Android. Il est basé sur IntelliJ IDEA et utilise le moteur de production Gradle. Il peut être téléchargé sous les systèmes d'exploitation Windows, macOS, Chrome OS et Linux.
\newline Pour en savoir plus, veillez
visiter le lien : \href{https://en.wikipedia.org/wiki/AndroidStudio.}{https://en.wikipedia.org/wiki/AndroidStudio.}
\subsection{Git}
\begin{comment}
	
\begin{figure}[h]
	\includegraphics[scale=0.3]{./Template LaTeX/Images/Git-Icon.png}
	\centering
	\caption{Git}
\end{figure}
\end{comment}
Git est un système de contrôle de version distribué gratuit et Open Source conçu pour gérer des
petits projets aux très grands projets avec rapidité et efficacité. C'est un logiciel libre créé par
Linus Torvalds, auteur du noyau Linux, et distribué selon les termes de la licence publique
générale GNU version 2 \href{https://edu.casio.com/support/fr/gplv2.html}{GPLv2}. Depuis 2016, il s’agit du logiciel de gestion de versions le
plus populaire. Git possède deux structures de données : une base d'objets et un cache de
répertoires. Il existe quatre types d'objets :
\begin{itemize}[label=$\ast$]
	
	\item  l'objet \textbf{blob} (pour binary large object désignant un ensemble de données brutes) qui
	représente le contenu d'un fichier ;
	\item l'objet \textbf{tree} (mot anglais signifiant arbre) qui décrit une arborescence de fichiers. Il est
	constitué d'une liste d'objets de type blobs et des informations qui leur sont associées
	telles que le nom du fichier et les permissions. Il peut contenir récursivement d'autres
	objets trees pour représenter les sous-répertoires ;
	\item l'objet \textbf{commit} (résultat de l'opération du même nom signifiant « valider une transaction
	») qui correspond à une arborescence de fichiers (tree) enrichie de métadonnées comme
	un message de description, le nom de l'auteur, etc. Il pointe également vers un ou
	plusieurs objets commit parents pour former un graphe d'historiques ;
	\item l'objet \textbf{tag} qui est une manière de nommer arbitrairement un commit
	spécifique pour l'identifier plus facilement. Il est en général utilisé pour marquer certains
	commits.
	
\end{itemize}


veillez
visiter le lien :
\href{https://git-scm.com/s}{https://git-scm.com/s}
\subsection{Le SGBD Firebase}
\begin{comment}
	
\begin{figure}[h]
	\includegraphics[scale=0.1]{./Template LaTeX/Images/firebase.png}
	\centering
	\caption{Firebase}
\end{figure}

\end{comment}
Firebase est une plateforme d’hébergement pour n’importe quel type d’application (mobiles,
Web, etc.) qui fournit aux développeurs une pléthore d'outils et de services pour les aider à
développer des applications de haute qualité, à élargir leur base d'utilisateurs et à générer
davantage de profits. Il propose d'héberger en NoSQL et en temps réel des bases de données,
du contenu, de l'authentification sociale (Google, Facebook, Twitter et Github) et des notifications, ou encore des services tels que par exemple un serveur de communication en
temps réel [1].
Lancé en 2011 sous le nom d'Envolve, par Andrew Lee et par James Templin, le service est
racheté par Google en octobre 2014. Il appartient aujourd'hui à la maison mère de Google :
Alphabet (maison-mère de groupe qui regroupe une majorité des activités du groupe comme
YouTube, Android, la recherche et les publicités). L'objectif premier de Firebase est de vous
libérer de la complexité de création et de la maintenance d'une architecture serveur, tout en vous
garantissant une scalabilité (capacité de l’application à s'adapter à un changement d'ordre de
grandeur de la demande, en particulier sa capacité à maintenir ses fonctionnalités et ses
performances en cas de forte demande) à toute épreuve (plusieurs milliards d'utilisateurs) et une
simplicité dans l'utilisation. Pour cela, Firebase a été décomposée en plusieurs produits
extrêmement riches et adaptés au monde du mobile, dont la liste suivante :
%\setcounter{secnumdepth}{5}
\subsubsection{Firebase Realtime Database :}
 Firebase Realtime Database est une base de données NoSQL hébergée dans le cloud qui vous
permet de stocker et de synchroniser des données entre vos utilisateurs en temps réel [1], [2].
Realtime Database est vraiment juste un gros objet JSON que les développeurs peuvent gérer
en temps réel.
La synchronisation en temps réel permet à vos utilisateurs d'accéder facilement à leurs données
depuis n'importe quel appareil, que ce soit sur le Web ou sur un appareil mobile. La base de
données en temps réel permet également à vos utilisateurs de collaborer les uns avec les autres.
Un autre gros avantage de Realtime Database est qu'elle est livrée avec des SDK mobiles et
Web, vous permettant de créer vos applications sans avoir besoin de serveurs. Lorsque vos
utilisateurs sont hors ligne, les SDK de base de données en temps réel utilisent le cache local
sur l'appareil pour servir et stocker les modifications. Lorsque l'appareil est en ligne, les données
locales sont automatiquement synchronisées.
\subsubsection{Firebase Authentication :}
Firebase Authentication fournit des services backend, des SDK faciles à utiliser et des
bibliothèques d'interfaces utilisateur prêtes à l'emploi pour authentifier les utilisateurs de votre
application [1], [2].
Vous pouvez authentifier les utilisateurs de votre application à l'aide des méthodes suivantes :
\begin{itemize}[label=$\ast$]
	\item Email et mot de passe
	\item Numéro de téléphone
	\item Compte Google
	\item Compte Facebook
	\item Compte Twitter
	\item Etc.
\end{itemize}
L'utilisation de Firebase Authentication facilite la création de systèmes d'authentification
sécurisés, tout en améliorant l'expérience de connexion et d'intégration pour les utilisateurs
finaux.
\subsubsection{Firebase Cloud Messaging (FCM)}
Firebase Cloud Messaging (FCM) fournit une connexion fiable et à faible consommation de
batterie entre votre serveur et vos périphériques, vous permettant d'envoyer et de recevoir
gratuitement des messages et des notifications sur iOS, Android et sur le Web [1], [2]. Vous
pouvez envoyer des messages de notification (limite de 2 Ko) et des messages de données
(limite de 4 Ko).
En utilisant FCM, vous pouvez facilement cibler les messages en utilisant des segments
prédéfinis ou créer les vôtres, en utilisant les données démographiques et comportementales.
Vous pouvez envoyer des messages à un groupe d'appareils abonnés à des rubriques
spécifiques, ou vous pouvez obtenir des informations aussi détaillées qu'un seul appareil.
FCM peut envoyer des messages instantanément, ou à un moment ultérieur dans le fuseau
horaire local de l'utilisateur. Vous pouvez envoyer des données d'application personnalisées,
telles que la définition des priorités, des sons et des dates d'expiration, ainsi que le suivi des
événements de conversion.
\subsubsection{Cloud Firestore}
Cloud Firestore est une base de données de documents NoSQL qui vous permet de facilement
stocker, synchroniser et interroger des données pour vos applications mobiles et Web à l'échelle
mondiale [1], [2]. Bien que cela puisse ressembler à quelque chose de similaire à la base de
données en temps réel, Firestore apporte beaucoup de nouvelles choses à la plateforme qui est
en fait quelque chose de complètement différent de Realtime Database.
Là où Realtime Database stocke des données sous la forme d'un arbre JSON géant, Cloud
Firestore adopte une approche beaucoup plus structurée. Firestore conserve ses données dans des objets appelés documents. Ces documents sont constitués de paires clé-valeur et peuvent
contenir n'importe quel type de données, depuis les chaînes jusqu'aux données binaires en
passant par des objets qui ressemblent à des arbres JSON (Firestore l'appelle des maps). Les
documents, à leur tour, sont regroupés en collections.
La base de données Firestore peut se composer de plusieurs collections qui peuvent contenir
des documents pointant vers des sous-collections. Ces sous-collections peuvent à nouveau
contenir des documents qui pointent vers d'autres sous-collections, et ainsi de suite.
Vous pouvez créer des hiérarchies pour stocker les données associées et récupérer facilement
les données dont vous avez besoin à l'aide de requêtes. Toutes les requêtes peuvent évoluer en
fonction de la taille de votre jeu de résultats. Votre application est donc prête à évoluer depuis
le premier jour.
\subsection{SGBD MariaDB}
\begin{comment}

\begin{figure}[h]
	\includegraphics[scale=0.5]{./Template LaTeX/Images/logo-mariadb-500px.png}
	\centering
	\caption{SGBD MariaDB}
\end{figure}
\end{comment}
Un gestionnaire de base de données libre.
ce projet est assurée par la fondation Maria DB, et sa
maintenance par la société Monty Program AB, créateur du
projet, il confère au logiciel l'assurance de rester libre. MariaDB a
plusieurs et différentes versions. Ells s'articulent sur le code
source de MySQL de la version 5.1 aux versions plus récentes
(comme la 5.6 fin 2012). Un serveur qui stocke les données dans
des tables séparées plutôt que de tout rassembler dans une seule
table. Du coup il améliore la rapidité et la souplesse de
l'ensemble.

\subsection{Logiciel de modélisation: Modelio}
\begin{comment}

\begin{figure}[h]
	\includegraphics[scale=0.5]{./Template LaTeX/Images/modelio.jpg}
	\centering
	\caption{Modelio}
\end{figure}
\end{comment}


Modelio est un logiciel open source et multiplateforme permettant, entre autres, la modélisation
UML et Business Process Model and Notation (BPMN). Pour en savoir plus, veuillez visiter le lien
\href{https://www.modelio.org/about-modelio/features.html.}{https://www.modelio.org/about-modelio/features.html.}
Sans doute, les logiciels de modélisation UML sont nombreux, à savoir, Visual Paradigm, Eclipse
Papyrus, StarUML, PowerDesigner, Umbrello, etc. Vu que les diagrammes UML que nous voulons
réaliser sont disponibles dans tous ces logiciels, il n’y avait pas en effet un choix à argumenter car
tous les choix étaient satisfaisants. Mais, de façon subjective, nous pouvons préciser que l’avantage de
Modelio dans notre contexte est le fait que je m’y suis déjà habitués. Les diagrammes que nous avons
réalisés avec Modelio sont ceux mentionnés ci-après.


\subsection{Motifs d’architecture logicielle : MVC}
%\subsubsection{MVC \newline}
Le patron de conception MVC est l’une des bases du Framework
laravel, c’est-à-dire modèle-vue- contrôleur. L’interaction avec la
base de données est assurée par le Model, les regroupe, traite et
gère les données.
Pour faire afficher ce que le modèle renvoie on fait appelle a la vue .
elle s’occupe d’autre part de la réception de toute interaction de
l’utilisateur. Ce sont ces actions-là que le contrôleur gère.
Ainsi que l’échange entre le modèle et la vue. Il intercepte toutes les
activités de utilisateur et, en fonction de ces activités, il actionne les
changements à effectuer sur l'application.
Les composants sont séparés en ces trois catégories précédentes
permet une clarté de architecture des dossiers et simplifie
grandement la tâche aux développeurs.
\begin{figure}[h!]
	\includegraphics[width=15cm, height=6cm]{./Template LaTeX/Images/Modèle-vue-contrôleur_(MVC)_-_fr.png}
	\caption{MVC}
	\label{fig:birds}
\end{figure}
\newline \newline \newline
\textbf{Les fichiers sont organisés comme suit :}
\begin{enumerate}
	\item [-] \textbf{Route (Dispatcher) : }il contient les définitions des chemins d’entrées pour l’utilisateur,
	autrement dit les URI possibles et les dirige sur la classe définit dans
	le contrôleur qui doit traiter l’information.
	\item [-] \textbf{ Modèle : }pour chaque table de notre base de données que l’on veut
	utiliser pour notre application, il faut créer un modèle pour chacun.
	Ainsi nous avons ici un modèle de notre application. Il permet de
	décrire la méthode d’accès aux données de la base, tous cela à
	travers un objet définit par ORM Eloquent(Object-Relational
	Mapping).
	\item [-] \textbf{Contrôleur : } il permet de récupérer les informations du modèle et de l’envoyer
	vers la vue pour la mise en forme.
	\item [-] \textbf{Vue : } la vue réceptionne la réponse qui est envoyée par le
	contrôleur .\newline
\end{enumerate}
\subsection{Serveur Web : Amazon Web Services (AWS)}
\begin{comment}

\begin{figure}[h]
	\includegraphics[scale=0.3]{./Template LaTeX/Images/512px-Amazon_Web_Services_Logo.svg.png}
	\centering
	\caption{Amazon Web Services (AWS)}
\end{figure}
\end{comment}

Amazon Web Services (AWS) est la plateforme cloud la plus complète et la plus largement adoptée au monde. Elle propose plus de 200 services complets issus de centres de données du monde entier. Des millions de clients (dont certaines des startups les plus dynamiques au monde, de très grandes entreprises et des agences fédérales de premier plan) utilisent AWS pour réduire leurs coûts, gagner en agilité et innover plus rapidement.

%\subsection{Conception}
\begin{comment}
	\subsubsection{Langage de modélisation : UML}
	\begin{figure}[h]
		\includegraphics[scale=1]{./Template LaTeX/Images/formation-uml-analser-concevoir.png}
		\centering
		\caption{UML}
	\end{figure}
	On a utilisé UML comme langage de modélisation.
	Langage de modélisation unifié UML (Unified modeling Langage) un
	consiste a modéliser une application logicielle d'une façon standard
	dans le cadre de conception orientée objet.
	UML consiste a couvrir le cycle de vie d'un logiciel depuis la
	spécification des besoins jusqu'au codage en offrant plusieurs
	moyens de description et de modélisation des acteurs.

\end{comment}




\section{Choix des outils de travail }
\subsection{Langages utilisés}
\subsubsection{Dart}
\begin{comment}
	
\begin{figure}[h]
	\includegraphics[scale=0.4]{./Template LaTeX/Images/Dart_programming_language_logo.svg.png}
	\centering
	\caption{Dart}
\end{figure}
\end{comment}

Dart est un langage de programmation open source à usage général. Il est initialement développé par Google. Dart est un langage orienté objet avec une syntaxe de C-style. Il prend en chargent les concepts de programmation tels que les interfaces, les classes, contrairement aux autres langages de programmation, Dart ne prend pas en charge les tableaux.
Les collections Dart peuvent être utilisées pour répliquer des structures de données telles que des tableaux, des génériques et un typage facultatif.
\newline Pour en savoir plus, veillez
visiter le lien : \href{https://www.tutorialspoint.com/flutter/flutter_introduction_to_dart_programming.htm}{https://www.tutorialspoint.com/flutter/dart}
\subsubsection{PHP}
\begin{comment}

\begin{figure}[h]
	\includegraphics[scale=0.3]{./Template LaTeX/Images/PHP-logo.svg.png}
	\centering
	\caption{PHP}
\end{figure}

\end{comment}

PHP est un langage de script utilisé le plus souvent côté serveur : dans cette architecture, le serveur interprète le code PHP des pages web demandées et génère du code (HTML, XHTML, CSS par exemple) et des données (JPEG, GIF, PNG par exemple) pouvant être interprétés et rendus par un navigateur web. PHP peut également générer d'autres formats comme le WML, le SVG et le PDF.

Il a été conçu pour permettre la création d'applications dynamiques, le plus souvent développées pour le Web. PHP est le plus souvent couplé à un serveur Apache bien qu'il puisse être installé sur la plupart des serveurs HTTP tels que IIS ou nginx. Ce couplage permet de récupérer des informations issues d'une base de données, d'un système de fichiers (contenu de fichiers et de l'arborescence) ou plus simplement des données envoyées par le navigateur afin d'être interprétées ou stockées pour une utilisation ultérieure.
\newline Pour en savoir plus, veillez
visiter le lien : \href{https://fr.wikipedia.org/wiki/PHP}{https://fr.wikipedia.org/wiki/PHP}
\subsubsection{Langage de modélisation : UML}
\begin{comment}

\begin{figure}[h]
	\includegraphics[scale=1]{./Template LaTeX/Images/formation-uml-analser-concevoir.png}
	\centering
	\caption{UML}
\end{figure}
\end{comment}

On a utilisé UML comme langage de modélisation.
Langage de modélisation unifié UML (Unified modeling Langage) un
consiste a modéliser une application logicielle d'une façon standard
dans le cadre de conception orientée objet.
UML consiste a couvrir le cycle de vie d'un logiciel depuis la
spécification des besoins jusqu'au codage en offrant plusieurs
moyens de description et de modélisation des acteurs.
\subsection{Frameworks utilisés}

\subsubsection{Flutter}
\begin{comment}

\begin{figure}[h]
	\includegraphics[scale=0.3]{./Template LaTeX/Images/Flutter.png}
	\centering
	\caption{Flutter}
\end{figure}
\end{comment}

Flutter est un framework de développement d’applications mobiles open source de Google. La principale raison de sa popularité est qu’il prend en charge la création 		   
d’applications multiplateformes. Flutter est également utilisé pour créer des apps interactives qui s’exécutent sur des pages web ou sur le bureau.\newline\newline
\textbf {- Les caractéristiques de Flutter :}
\begin{enumerate}
	\item Base de code unique pour Android et iOS
	\item Fonction de rechargement à chaud (hot reload)
	\item Open-source et par Google
	\item Programmation Dart \newline
\end{enumerate}
\textbf {- Architecture d’une application Flutter :} 
\begin{comment}
	content...

\begin{enumerate}
	\item[•] Flutter a une architecture modulaire en couches. Cela vous permet d’écrire votre logique d’application une seule fois et d’avoir un comportement cohérent sur toutes les plates-formes, même si le code du moteur sous-jacent diffère selon la plate-forme.
	\item[•] L’architecture en couches expose également différents points de personnalisation et de remplacement, si nécessaire.
	\item[•] Le concept de base du framework Flutter est dans Flutter, tout est un widget. Widget sont essentiellement des composants d’interface utilisateur utilisés pour créer l’interface utilisateur de l’application.
	\item[•] L’interface utilisateur est construite à l’aide d’un ou plusieurs enfants (widgets), qui se construisent à nouveau à l’aide de ses enfants
	widgets.
	\item[•] Cette fonctionnalité de composabilité nous aide à créer une interface utilisateur de toute complexité.
	\item[•] Flutter a une architecture modulaire qui effectivement permet d’écrire le code une seule fois et de l’utiliser sur plusieurs plateformes (Android, ios, ou web) même si le moteur est différent de plateforme à plateforme, pour le développeur c’est transparent. voici en image cette architecture:
\end{enumerate}
\end{comment}
Flutter a une architecture modulaire qui effectivement permet d’écrire le code une seule fois et de l’utiliser sur plusieurs plateformes (Android, ios, ou web) même si le moteur est différent de plateforme à plateforme, pour le développeur c’est transparent. voici en image cette architecture:


\begin{figure}[h!]
	\includegraphics[width=450px,height=250px]{./Template LaTeX/Images/Architecture-dune-application-Flutter-1.jpg}
	\caption{Architecture d’une application Flutter.}
	\label{fig:birds}
\end{figure}

	
	 L’architecture Flutter comprend principalement quatre composants:
	\begin{enumerate}
		\item[•] \textbf{Flutter Engine (Moteur de flutter): }
		Le Flutter Engine est un environnement d’exécution portable pour les applications mobiles de haute qualité. Il implémente les bibliothèques principales de Flutter, y compris l’animation et les graphiques, les E/S de fichiers et de réseau, la prise en charge de l’accessibilité, l’architecture des plugins, ainsi qu’un environnement d’exécution et une chaîne d’outils Dart pour développer, compiler et exécuter des applications Flutter.
		\item[•] \textbf{Foundation Library (Bibliothèque de la Fondation):} Il contient tous les packages requis pour les éléments de base de l’écriture d’une application Flutter. Ces librairies sont écrites en langage Dart.
		\item[•] \textbf{Widgets (Widget) :}
		\begin{enumerate}
			\item[*] Les widgets sont les éléments de base de l’interface utilisateur dans Flutter.
			\item[*] La conception de l’interface utilisateur pour une application implique la composition et la modification de divers widgets simples tels que du texte, des formes et des animations pour en créer des plus complexes.
			\item[*] Les widgets Flutter ne sont que des éléments de votre interface utilisateur. Si vous êtes familier avec le développement Android ou iOS, vous établirez la connexion immédiate aux vues (pour Android) et aux UIViews (pour iOS). C’est une bonne comparaison à faire et vous ferez bien de commencer votre voyage avec cet état d’esprit. Une façon plus précise de penser, cependant, est qu’un widget est un plan
		\end{enumerate}	
		\item[•] \textbf{Embedder Platform Specific (widgets spécifiques) :}
		Embbeder est différent pour chaque plateforme et son rôle est de créer l’exécutable ou les modules pour chaque plateforme.
		\begin{enumerate}
			\item[*] Les applications Flutter sont packagées de la même manière que toute autre application native pour les plates-formes cibles.
			\item[*] Flutter fournit un certain nombre d’intégrateurs spécifiques à la plate-forme pour les plates-formes cibles qui coordonnent une application flottante avec le système d’exploitation sous-jacent pour accéder à des services tels que les surfaces de rendu, l’accessibilité, la saisie et la gestion de la boucle d’événements de message.
			\item[*] L’intégrateur est généralement écrit dans le meilleur langage approprié pour la cible. plate-forme.
			\item[*] L’embedder pour Android est écrit en C++ et Java, Objective C/Objective C++ pour iOS et macOS, et C++ pour Windows et Linux. Avec l’utilisation de l’embedder, le code Flutter peut également être intégré dans une application existante en tant que module.
		\end{enumerate}	
	\end{enumerate}
 Pour en savoir plus, veillez
visiter le lien : \href{https://apcpedagogie.com/architecture-dune-application-flutter/}{https://apcpedagogie.com/architecture-dune-application-flutter/}
\newpage
\subsubsection{Laravel}
\begin{comment}

\begin{figure}[h]
	\includegraphics[scale=0.2]{./Template LaTeX/Images/462px-Laravel.svg.png}
	\centering
	\caption{Laravel}
\end{figure}
\end{comment}
Laravel est un framework web open-source écrit en  \href{https://fr.wikipedia.org/wiki/PHP}{PHP} respectant le principe modèle-vue-contrôleur et entièrement développé en programmation orientée objet. Laravel est distribué sous \href{https://fr.wikipedia.org/wiki/Licence_MIT}{licence MIT}, avec ses sources hébergées sur 
\href{https://fr.wikipedia.org/wiki/GitHub}{GitHub}.\newline\newline
\textbf {- Architecture de Laravel :}

\begin{figure}[h!]
	\includegraphics[width=500px,height=200px]{./Template LaTeX/Images/Laravel-MVC-framework.jpg}
	\caption{Architecture de Laravel}
	\label{fig:birds}
\end{figure}
\newpage
\section{Implémentation}
\subsection{Etape de réalisation}
Pour réaliser notre système, il y a des étapes à élaborer dans l’ordre suivant :
\begin{itemize}[label=$\ast$]
	\item \textbf{Conception de la base de données :} La conception d’une base de données est la
	première étape. Le choix des algorithmes et de l’approche de travail exige l’utilisation
	d’une base de données spécifique (en fonction du système à développer).
	\item \textbf{Extraction des données :}  On va utiliser notre SGBD Firebase ainsi que notre base de
	données en interaction avec l’application pour extraire des informations. Firebase est
	conçu avec des fonctionnalités et des requêtes de sélection de données assez spécifique
	et facile à utiliser.
	\item \textbf{Conception et développement du front-end et du back-end :} 
	Cette étape consiste à
	détailler la conception coté client et coté serveur. Il s’agit de mettre en place un design
	ergonomique, simple et attractif répondant aux exigences du système. Le choix d’un
	serveur d’application adéquat aux fonctionnalités et aux données est une étape
	fondamentale pour le bon fonctionnement de l’application.
	\item \textbf{Développement de back-end :} On commence par le développement de l’application coté
	serveur, dans notre cas avec \textbf{Laravel}. C’est la partie du code exécuté sur le serveur afin
	de vérifier le comportement des fonctionnalités de base du système.
	\item \textbf{Développement front-end :} On développe la partie client en interaction avec le serveur.
	C’est la conception de l'interface graphique utilisateur. En effet, il s'agit de la partie
	visible de l'application, destinée à être manipulée par un tiers.
\end{itemize}
\subsection{Interfaces Homme/Machine}
Dans ce qui suit, nous présentons quelques  fonctionnalités avec leur interfaces \newline graphiques de notre produit final des applictions CADORIM et Cadorim service.
\subsubsection{Application CADORIM}
\begin{itemize}[label=$\ast$]
		\item \textbf{Première interface :} L'utilisateur du Cadorim avant d'être invite à consulter les services de l'application, doit 
			choisies leur langue avant qu'il pass a l'interface suivante.
			
			\begin{figure}[!ht]
				\centering
				\begin{subfigure}{0.3\textwidth}
					\includegraphics[width=\hsize, valign=m ]{./Template LaTeX/Images/1.jpg}
					\caption{Choix de langue}
					\label{fig.SICAPI}
				\end{subfigure}
				\qquad\tikz[baseline=-\baselineskip]\draw[ultra thick,->] (0,0) -- ++ (1,0);\qquad
				\begin{subfigure}{0.3\textwidth}
					\includegraphics[width=\hsize, valign=m]{./Template LaTeX/Images/2.jpg}
					\caption{Interface suivante}
					\label{fig.painel_sicapi}
				\end{subfigure}
				\caption{Première interface}
				\label{fig.sicapi}
			\end{figure}
		%%%%%%%%%%%%%%%%%%%%%%%%%%%%%%% authentificationb %%%%%%%%%%%%
		\newpage
		\item \textbf{L’interface
			d’authentification
			:} La figure suivante représente l’interface d’authentification de notre application. Elle permet aux utilisateurs de s’identifier en introduisant leurs identifiants afin d’accéder aux fonctionnalités de l’application.
			\begin{figure}%
				\centering
				{{\includegraphics[width=5cm]{./Template LaTeX/Images/3.jpg} }}%
				\caption{Interface d'authentification}%
				\label{fig:example}%
			\end{figure}
		\newpage
		La demande d’identification du client est traitée pour vérifier ses paramètres dans la base de
		données. L’absence de l’utilisateur dans la base de données ou une erreur de saisie des
		informations entraine une alerte d’erreur d’authentification.\newline
		%%%%%%%%%%%%%%%%%%%%%%%%%% Inscruire %%%%%%%%%%%%%%%%%%%%%%5
		\item \textbf{L’interface d’inscription :}
		Avant de pouvoir s’authentifier, l’utilisateur doit
		impérativement s’enregistrer au préalable dans la base de données. La figure suivante
		représente l’interface de création de compte pour un client.Toutes les informations personnelles sont extraites partir de scan de code MRZ.
		\begin{figure}[!ht]
			\centering
			\begin{subfigure}{0.3\textwidth}
				\includegraphics[width=\hsize, valign=m ]{./Template LaTeX/Images/4.jpg}
				\caption{Interface d’inscription}
				\label{fig.SICAPI}
			\end{subfigure}
			\qquad\tikz[baseline=-\baselineskip]\draw[ultra thick,->] (0,0) -- ++ (1,0);\qquad
			\begin{subfigure}{0.3\textwidth}
				\includegraphics[width=\hsize, valign=m]{./Template LaTeX/Images/23.jpg}
				\caption{Scanner le code MRZ}
				\label{fig.painel_sicapi}
			\end{subfigure}
	
		\begin{subfigure}{0.3\textwidth}
			\includegraphics[width=\hsize, valign=m]{./Template LaTeX/Images/24.jpg}
			\caption{Saisir le rest des données}
			\label{fig.painel_sicapi}
		\end{subfigure}
			\caption{Interface d’inscription}
			\label{fig.sicapi}
		\end{figure}
	
	%%%%%%%%%%%%%%%%%%%%%%%%%%%%%5 Home
	
	%%%%%%%%%%%%%%%%%%%%%%%%%%%%%%%%%%%%
%	\newpage
	\item \textbf{Processus de la transaction
		:} La figure ~\ref{ProcessusT} %suivante 
	montre les processus de la transaction depuis la sélection de montant jusqu'à la validation (une notification de succès va Apparence si le traitement se fait avec succès)
\begin{comment}
	content...

\begin{figure}[!ht]
	\centering
	\begin{subfigure}{0.3\textwidth}
		\includegraphics[width=\hsize, valign=m ]{./Template LaTeX/Images/5.jpg}
		\caption{Interfaces d'accueil}
		\label{fig.SICAPI}
	\end{subfigure}
	\qquad\tikz[baseline=-\baselineskip]\draw[ultra thick,->] (0,0) -- ++ (1,0);\qquad
	\begin{subfigure}{0.3\textwidth}
		\includegraphics[width=\hsize, valign=m]{./Template LaTeX/Images/11.jpg}
		\caption{Interface de transfert}
		\label{fig.painel_sicapi}
	\end{subfigure}
%%%%%%%%%%%%%%%%%%%%%%%%%%%%%
\begin{subfigure}{0.3\textwidth}
	\includegraphics[width=\hsize, valign=m ]{./Template LaTeX/Images/12.jpg}
	\caption{Choix de wilaya}
	\label{fig.SICAPI}
\end{subfigure}
\qquad\tikz[baseline=-\baselineskip]\draw[ultra thick,->] (0,0) -- ++ (1,0);\qquad
\begin{subfigure}{0.3\textwidth}
	\includegraphics[width=\hsize, valign=m]{./Template LaTeX/Images/13.jpg}
	\caption{Choix d'agence}
	\label{fig.painel_sicapi}
\end{subfigure}
%%%%%%%%%%%%%%%%%%%%%%%%%%%%%%%%%%%%%%%%%%%%%%%%%%%%%%%%%%%

	\caption{Interfaces d'accueil}
	\label{fig.sicapi}
\end{figure}
\end{comment}
\begin{figure}
	\centering
	\begin{subfigure}[b]{0.3\textwidth}
		\centering
		\includegraphics[width=\textwidth]{./Template LaTeX/Images/5.jpg}
		\caption{Interfaces d'accueil}
		\label{fig:y equals x}
	\end{subfigure}
	\hfill
	\begin{subfigure}[b]{0.3\textwidth}
		\centering
		\includegraphics[width=\textwidth]{./Template LaTeX/Images/11.jpg}
		\caption{Interface de transfert}
		\label{fig:three sin x}
	\end{subfigure}
	\hfill
	\begin{subfigure}[b]{0.3\textwidth}
		\centering
		\includegraphics[width=\textwidth]{./Template LaTeX/Images/12.jpg}
		\caption{Choix de wilaya}
		\label{fig:five over x}
	\end{subfigure}
	\newline
		\centering
	\begin{subfigure}[b]{0.3\textwidth}
		\centering
		\includegraphics[width=\textwidth]{./Template LaTeX/Images/13.jpg}
		\caption{Choix d'agence}
		\label{fig:y equals x}
	\end{subfigure}
	\hfill
	\begin{subfigure}[b]{0.3\textwidth}
		\centering
		\includegraphics[width=\textwidth]{./Template LaTeX/Images/14.jpg}
		\caption{Choix de catre bancaire}
		\label{fig:three sin x}
	\end{subfigure}
	\hfill
	\begin{subfigure}[b]{0.3\textwidth}
		\centering
		\includegraphics[width=\textwidth]{./Template LaTeX/Images/15.jpg}
		\caption{État de transfert}
		\label{fig:five over x}
	\end{subfigure}
	\caption{Processus de transactions}
	\label{ProcessusT}
\end{figure}
\newpage
\item \textbf{L’interface de transfert
	:}La figure~\ref{transfert} montre l'historique des transactions de l'utilisateur  vers les bénéficiaires avec détails pour chaque transaction.

\begin{figure}
	\centering
	\begin{subfigure}[b]{0.3\textwidth}
		\centering
		\includegraphics[width=\textwidth]{./Template LaTeX/Images/16.jpg}
		\caption{Transfere avec succés}
		\label{fig:y equals x}
	\end{subfigure}
	\hfill
	\begin{subfigure}[b]{0.3\textwidth}
		\centering
		\includegraphics[width=\textwidth]{./Template LaTeX/Images/17.jpg}
		\caption{Historique transactions}
		\label{fig:three sin x}
	\end{subfigure}
	\hfill
	\begin{subfigure}[b]{0.3\textwidth}
		\centering
		\includegraphics[width=\textwidth]{./Template LaTeX/Images/18.jpg}
		\caption{Détails de la transaction}
		\label{fig:five over x}
	\end{subfigure}
	\caption{Transferts}
	\label{transfert}
\end{figure}

\item \textbf{L’interface de discussion
	:} La figure suivante montre que le client peut envoyer un message (photo ou un texte) au service client et bien sur voir l'historique des messages.
	\begin{figure}%
	\centering
	{{\includegraphics[width=5cm]{./Template LaTeX/Images/c.jpg} }}%
	\caption{Interface de discussion}%
	\label{fig:example}%
\end{figure}
\newpage
\item \textbf{Profil de l'utilisateur:} L'utilisateur peut mettre à jour leur profil ou ajouter un mode paiement comme il montre la figure suivante
\begin{figure}
	\centering
	\begin{subfigure}[b]{0.3\textwidth}
		\centering
		\includegraphics[width=\textwidth]{./Template LaTeX/Images/6.jpg}
		\caption{Mon profil}
		\label{fig:y equals x}
	\end{subfigure}
	\hfill
	\begin{subfigure}[b]{0.3\textwidth}
		\centering
		\includegraphics[width=\textwidth]{./Template LaTeX/Images/7.jpg}
		\caption{Ajouter une carte}
		\label{fig:three sin x}
	\end{subfigure}
	\hfill
	\begin{subfigure}[b]{0.3\textwidth}
		\centering
		\includegraphics[width=\textwidth]{./Template LaTeX/Images/8.jpg}
		\caption{Importer une photo}
		\label{fig:five over x}
	\end{subfigure}
\newline
	\centering
\begin{subfigure}{0.3\textwidth}
	\includegraphics[width=\hsize, valign=m ]{./Template LaTeX/Images/9.jpg}
	\caption{État}
	\label{fig.SICAPI}
\end{subfigure}
%\qquad\tikz[baseline=-\baselineskip]\draw[ultra thick,->] (0,0) -- ++ (1,0);\qquad
\begin{subfigure}{0.3\textwidth}
	\includegraphics[width=\hsize, valign=m]{./Template LaTeX/Images/10.jpg}
	\caption{Interface suivante}
	\label{fig.painel_sicapi}
\end{subfigure}
	\caption{Profil de l'utilisateur
	}
	\label{fig:three graphs}
\end{figure}


%%%%%%%%%%%%%%%%%%%%%%%%%%%%%%%%MENU%%%%%%%%%%%%%%%%%%%%%%%%%%%%%%%
\newpage
\item \textbf{L’interface de menu
	:} À partir de cette fenêtre le client a l'accès aux autres fonctionalistes de l'application (Modifier mot de passe,changer la langue ....)
\begin{figure}%
	\centering
	{{\includegraphics[width=5cm]{./Template LaTeX/Images/menu.jpg} }}%
	\caption{Interface de menu}%
	\label{fig:example}%
\end{figure}
\end{itemize}

\subsubsection{Application ServiceClient}
\begin{itemize}[label=$\ast$]
		\item \textbf{L’interface
		d’authentification
		:} La figure suivante représente l’interface d’authentification de notre application. Elle permet aux utilisateurs de s’identifier en introduisant leurs identifiants afin d’accéder aux fonctionnalités de l’application.
	
	\begin{figure}%
		\centering
		{{\includegraphics[width=5cm]{./Template LaTeX/Images/From_emu/login.png} }}%
		\caption{Interface d'authentification}%
		\label{fig:example}%
	\end{figure}
\newpage
	\item \textbf{L’interface du compte administrateur
	:} La figure~\ref{Home} représente la page principale de l’application pour l'administrateur. C’est à partir de cette fenêtre qu’est accessible la majorité des fonctionnalités de base de l’application(Gérer les comptes de service client).
\begin{comment}
	content...
}
\begin{figure}
	\centering
	\begin{subfigure}[b]{0.3\textwidth}
		\centering
		\includegraphics[width=\textwidth]{./Template LaTeX/Images/6.jpg}
		\caption{Interfaces d’accueil}
		\label{fig:y equals x}
	\end{subfigure}
	\hfill
	\begin{subfigure}[b]{0.3\textwidth}
		\centering
		\includegraphics[width=\textwidth]{./Template LaTeX/Images/7.jpg}
		\caption{Supprimer un compte}
		\label{fig:three sin x}
	\end{subfigure}
	\hfill
	\begin{subfigure}[b]{0.3\textwidth}
		\centering
		\includegraphics[width=\textwidth]{./Template LaTeX/Images/8.jpg}
		\caption{Mettre à jour un compte}
		\label{fig:five over x}
	\end{subfigure}
	\caption{Interface du compte principal
}
\label{Home}
\end{figure}
\end{comment}
%%%%%%%%%%%%%%%%%%%%%%%%%%%%%%%%%%%%%%%%%%%%%%%%%%%%%%%%%%%%%
\begin{figure}
	\centering
	\begin{subfigure}[b]{0.3\textwidth}
		\centering
		\includegraphics[width=\textwidth]{./Template LaTeX/Images/From_emu/A_home.png}
		\caption{Interfaces d’accueil}
		\label{fig:y equals x}
	\end{subfigure}
	\hfill
	\begin{subfigure}[b]{0.3\textwidth}
		\centering
		\includegraphics[width=\textwidth]{./Template LaTeX/Images/From_emu/A_update_modif.png}
		\caption{Gérer un compte}
		\label{fig:three sin x}
	\end{subfigure}
	\hfill
	\begin{subfigure}[b]{0.3\textwidth}
		\centering
		\includegraphics[width=\textwidth]{./Template LaTeX/Images/From_emu/A_up_s.png}
		\caption{Mettre à jour un compte}
		\label{fig:five over x}
	\end{subfigure}
	\newline
	\centering
	\begin{subfigure}{0.3\textwidth}
		\includegraphics[width=\hsize, valign=m ]{./Template LaTeX/Images/From_emu/A_add_s.png}
		\caption{Ajouter un compte}
		\label{fig.SICAPI}
	\end{subfigure}
	%\qquad\tikz[baseline=-\baselineskip]\draw[ultra thick,->] (0,0) -- ++ (1,0);\qquad
	
	\caption{Interface du compte administrateur
}
\label{Home}
\end{figure}
%%%%%%%%%%%%%%%%%%%%%%%%%%%%%%%%%%%%%%%%%%%%%%%%%%%%%%%%%%%%%
\newpage
\item \textbf{L’interface du compte service client
	:} 
La figure~\ref{serviceCl} représente la page principale de l’application pour le compte service client C’est à partir de cette fenêtre l'utilisateur peut répondre aux clients.
\begin{figure}
	\centering
\begin{subfigure}{0.3\textwidth}
	\includegraphics[width=\hsize, valign=m ]{./Template LaTeX/Images/From_emu/a.png}
	\caption{Notification}
	\label{klk}
\end{subfigure}
\begin{subfigure}{0.3\textwidth}
	\includegraphics[width=\hsize, valign=m ]{./Template LaTeX/Images/From_emu/no_vue.png}
	\caption{Interface d’accueil}
	\label{klk}
\end{subfigure}
\begin{subfigure}{0.3\textwidth}
	\includegraphics[width=\hsize, valign=m ]{./Template LaTeX/Images/From_emu/b.png}
	\caption{Interface de discussion}
	\label{klk}
\end{subfigure}
	\caption{Interface du compte service client
}
\label{serviceCl}
\end{figure}
\end{itemize}



	\chapter*{Conclusion}
\addcontentsline{toc}{chapter}{Conclusion}
Dans ce chapitre, nous avons décrit brièvement le processus de développement de notre application
de transfert d'argent en spécifiant l’environnement de développement, le choix des
outils ainsi que la démarche suivie pour la réalisation.
En effet, nous avons achevé l’implémentation tout en respectant la conception, mais les tests
des différents cas d’utilisations sont toujours en cours envie d’une amélioration de l’application
avant sa publication.
En d’autres termes, nous avons la version beta (test et amélioration avant sa publication) de
l’application installée dans notre environnement de développement. Aussi, nous avons prévu la
période pendant laquelle la solution finale sera déployée les plateformes de téléchargement.
%	\chapter{Conclusion générale et perspectives}
	\chapter*{Bibliographie}
\addcontentsline{toc}{chapter}{Bibliographie}
\begin{enumerate}
	\item[[ 1]] N. Symth, Firebase Essentials. Cary: Payload Media, 2017.
	\item[[ 2]] J. Crowther, Firebase. London: Constable, 2015.
	\item[[ 3]] S. Madise, The regulation of mobile money. New York, NY: Springer Berlin Heidelberg,
	2019.
	\item[[ 4]] E. M. Ndadoum et B. Kordjé, Mobile money en Afrique - Son rôle pour l’inclusion
	financière au Tchad. L’Harmattan, 2020.
	\item[[ 5]] O. Fédior, « Mobile money/Mobile banking : La guerre des transferts - OSIRIS :
	Observatoire sur les Systèmes d’Information, les Réseaux et les Inforoutes au Sénégal »,
	mars 08, 2019. http://www.osiris.sn/Mobile-money-Mobile-banking-La.html (consulté le
	juill. 27, 2020).
	\item[[ 6]] B. A. Lassaad, « Cameroun: pénurie inédite de pièces de monnaie. », janv. 10, 2020.
	https://www.aa.com.tr/fr/afrique/cameroun-pénurie-inédite-de-pièces-de-
	monnaie/1698715 (consulté le juin 13, 2020).
	
	
	\item[[ 7]] Glabb, Ryan et al. (2007), « Multi-mode operator for SHA-2 hash functions », in : journal of
	systems architecture 53.2-3, p. 127–138.
	\item[[ 8]] Dworkin, Morris J (2015), SHA-3 standard : Permutation-based hash and extendable-output
	functions, rapp. tech.
	\item[[ 9]] Dolmatov, Vasily et Alexey Degtyarev (2013), « GOST R 34.11-2012 : hash function », in :
	Independent Submission, Ed. Request for Comments : Updates 5831, p. 2070–1721.
	
\end{enumerate}
	%\chapter{Implémentation}
Durant la réalisation de ce projet, nous avons essayé d’utiliser différents
outils de développement, d’une part afin de rendre la tâche de la
réalisation plus facile, d’autre part pour que notre système soit robuste et
répond parfaitement a nos besoins , et que nos interfaces soient claires et
faciles à utiliser.
\section{Architecture de l'application}
Cadorim est une application embarquée qui se connect à un serveur de base de données distant, via Internet, afin de récupérer les données, Ce qui necessite aussi l'intégration d'un serveur web entre l'application client et le serveur de bases de données.
D'où larchitecture de notre application est à 3 niveaux, elle est partagée entre;
\begin{enumerate}
	\item \textbf {L'application mobile (IOS ou Android) : }  Ce le client qui demande les ressources.
	\item \textbf{Le Serveur Web :} Vue que les données serons communiquées entre deux environnements hétérogènes, le rôle principale du serveur web est de gérer la communication entre le client (Android ou IOS) et le serveur de données.
	\item \textbf{Le serveur de base de données:} fournis les données au serveur web.
\end{enumerate}

\begin{figure}[h!]
	\includegraphics[width=17cm, height=10cm]{./Template LaTeX/Images/architect_after_edit.png}
	\caption{Architecture logicielle de l’application.}
	\label{fig:birds}
	
	
\end{figure}
\newpage
\section{Interfaces graphiques}
Les interfaces utilisateur doivent respecter les heuristiques d'utilité pour permettre à l'utilisateur un accès facile à ces interfaces afin de garantir une bonne compréhension des fonctionnalités de l'application. Nous présentons ici les interfaces les plus significatives de l'application.
\subsection{Interfaces d'acceil}

L'utilisateur du Cadorim avant d'être invite à consulter les services de l'application, doit 
choisies leur lange avant qu'il pass a l'interface suivante.

\begin{figure}[!ht]
	\centering
	\begin{subfigure}{0.3\textwidth}
		\includegraphics[width=\hsize, valign=m]{./Template LaTeX/Images/1.jpg}
		\caption{Choix de lange}
		\label{fig.SICAPI}
	\end{subfigure}
	\qquad\tikz[baseline=-\baselineskip]\draw[ultra thick,->] (0,0) -- ++ (1,0);\qquad
	\begin{subfigure}{0.3\textwidth}
		\includegraphics[width=\hsize, valign=m]{./Template LaTeX/Images/2.jpg}
		\caption{Interfaces Accueil}
		\label{fig.painel_sicapi}
	\end{subfigure}
	\caption{Interfaces d'acceil}
	\label{fig.sicapi}
\end{figure}
\begin{comment}
	
\begin{figure}%
	\centering
	\subfloat[\centering choix de lange ]{{\includegraphics[width=5cm]{./Template LaTeX/Images/1.jpg} }}%
	\qquad	
	\subfloat[\centering  Acceuil]{{\includegraphics[width=5cm]{./Template LaTeX/Images/2.jpg} }}%
	\caption{Interfaces Accueil}%
	\label{fig:example}%
\end{figure}


	content...

\begin{figure}%
	\centering
	\subfloat[\centering choix de lange ]{{\includegraphics[width=5cm]{./Template LaTeX/Images/3.jpg} }}%
	\qquad
	\subfloat[\centering  Acceuil]{{\includegraphics[width=5cm]{./Template LaTeX/Images/4.jpg} }}%
	\caption{Interfaces Accueil}%
	\label{fig:example}%
\end{figure}
\end{comment}

	%\chapter{Interface et utilisation de l'application}
\label{sec:Interface}	
\end{document}