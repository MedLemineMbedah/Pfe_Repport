\PassOptionsToPackage{natbib}{biblatex}
\documentclass{bredele}
\addbibresource{references.bib}

\usepackage{makecell, amssymb, amsthm, mathtools, rotating, colortbl, enumitem, nomencl, latexsym, bookmark, url, subcaption, float, multirow, algorithm, algorithmic, color}
\usepackage{arabtex, utf8}
\setcode{utf8}
\usepackage[nomain, acronym]{glossaries}

\makeglossaries


\definecolor{Skyblue}{rgb}{0.15,0.4,0.65}

\floatstyle{plaintop}
\restylefloat{table}

\newtheoremstyle{mystyle}%             % Name
{}%                                    % Space above
{}%                                    % Space below
{\itshape}%                            % Body font
{}%                                    % Indent amount
{\bfseries}%                           % Theorem head font
{}%                                    % Punctuation after theorem head
{\newline}%                            % Space after theorem head, ' ', or \newline
{}%                                    % Theorem head spec (can be left empty, meaning `normal')

\theoremstyle{mystyle}

\addto\captionsfrench{\def\tablename{Tableau}}
\setlist[itemize]{label=\textbullet}
\setcounter{secnumdepth}{3}

\pagenumbering{Roman}

\makeatletter
\newcommand{\closenomencl}{%
	\closeout\@nomenclaturefile%
}
\makeatother

\newcommand{\writenomencl}[1]{%
	\closenomencl%
	\IfFileExists{#1.nlo}{%
		\write18{%
			makeindex -s nomencl.ist -o #1.nls -t #1.nlg #1.nlo%
		}% 
	}{\typeout{Nothing there}}%
}

\AtEndDocument{\writenomencl{\jobname}}

% Remove " CHAPTER 0 " from header
\renewcommand{\chaptermark}[1]{%
	\ifnum\value{chapter}>0
	\markboth{Chapitre \thechapter{}: #1}{}%
	\else
	\markboth{#1}{}%
	\fi}

\title{\textbf{\textcolor{Skyblue}{Conception et simulation d'un système d'échange de clés S-SEJAD}}}

\discipline{Informatique}

\jury{
	\begin{description}		
		\item[Président~:]~~~~~~~~~~~~~~ Hamidou \textsc{Dathe}, Professeur, Université Cheikh Anta Diop
		\item[Rapporteurs~:]~~~~~~~~~ .... \textsc{....}, .................................
		\item[]~~~~~~~~~~~~~~~~~~~~~~~~~~~~~~ ....  \textsc{....}, .................................	
		\item[Examinateurs~:]~~~~~~~ .... \textsc{....}, .................................
		\item[]~~~~~~~~~~~~~~~~~~~~~~~~~~~~~~ .... \textsc{....}, .................................
		\item[Co-endadrant~:]\ Khadidiatou   \textsc{WANE Keita},Maître de Conférences , Université Cheikh Anta Diop
		\item[Directeur de thèse~:]\ Khaly  \textsc{TALL}, Professeur, Université Cheikh Anta Diop
	\end{description}
}

\logouniversite{Images/ISCAE}
\scalelogouniversite{0.3}

\unite{\large Présenté par\\ Mohamed Saleck Lebchir}
\ecoledoc{Pour obtenir le diplôme de\\Master Professionnel de l'ISCAE}


\begin{document}
	
	\maketitle
	
	\tableofcontents
	\listoffigures
	\let\cleardoublepage\clearpage
	\listoftables
	
	\newacronym{UML}{UML}{Unified Modeling Language.}
	\newacronym{API}{API}{Application Programming Interface.}
	\newacronym{DAO}{DAO}{Data Access Object.}
	\newacronym{IDE}{IDE}{Integrated Development Environment (Environnement de Développement Intégré).}
	\newacronym{HTTP}{HTTP}{HyperText Transfer Protocol.}
	\newacronym{HTTPS}{HTTPS}{HyperText Transfer Protocol Secure.}
	\newacronym{JSON}{JSON}{JavaScript Object Notation.}
	\newacronym{MERISE}{MERISE}{Méthode d’Étude et de Réalisation Informatique par les Sous-Ensembles ou pour les Systèmes d’Entreprise.}
	\newacronym{MVC}{MVC}{Model-View-Controller.}
	\newacronym{SGBD}{SGBD}{Système de Gestion de Bases de Données.}
	\newacronym{SGBDR}{SGBDR}{Système de Gestion de Bases de Données Relationnelle.}
	\newacronym{OTP}{OPT}{One-Time Password}
	\newacronym{SOA}{SOA}{Service Oriented Architecture}
	\newacronym{ORM}{ORM}{Object-Relational Mapping (en français Mapping Objet-Relationnel)}
	\newacronym{APK}{APK}{Android Package Kit}
	\newacronym{AAB}{AAB}{Android App Bundles}
	
	\printacronyms
	\printglossary[title={List of Abbreviations}]

	\let\cleardoublepage\clearpage

\chapter{Introduction générale}
\label{sec:DescriptionDuProjet}

Savoir qui est votre client et adopter des protocoles pour prévenir la criminalité financière sont des défis permanents pour les institutions financières. De manière significative, les institutions financières (y compris les banques, les coopératives de crédit et les sociétés financières du Fortune 50) doivent se conformer à un ensemble des réglementations de plus en plus complexes pour la vérification de l'identité des clients appelée KYC.

KYC, également connu sous le nom de "Know Your Customer" ou "Know Your Client", est un ensemble de procédures permettant de vérifier l'identité d'un client avant ou pendant les transactions avec les banques et autres institutions financières. Le respect des réglementations KYC peut aider à tenir à distance le blanchiment d'argent, le financement du terrorisme et d'autres stratagèmes de fraude courants. En vérifiant d'abord l'identité et les intentions d'un client au moment de l'ouverture du compte, puis en comprenant ses habitudes de transaction, les institutions financières sont en mesure d'identifier plus précisément les activités suspectes. 

Les institutions financières sont soumises à des normes de plus en plus strictes en matière de lois KYC. Ils doivent dépenser plus d'argent pour se conformer à KYC ou être passibles de lourdes amendes. Ces réglementations signifient que presque toutes les entreprises, plateformes ou organisations qui interagissent avec une institution financière pour ouvrir un compte ou effectuer des transactions devront se conformer à ces obligations.

La gestion de la relation client (CRM) est la combinaison de pratiques, de stratégies et de technologies que les entreprises utilisent pour gérer et analyser les interactions et les données client tout au long du cycle de vie du client. L'objectif est d'améliorer les relations de service client, de contribuer à la fidélisation de la clientèle et de stimuler la croissance des ventes. Les systèmes CRM compilent les données client à travers différents canaux, ou points de contact, entre le client et l'entreprise, qui peuvent inclure le site Web de l'entreprise, le téléphone, le chat en direct, le publipostage, les supports marketing et les réseaux sociaux. Les systèmes CRM peuvent également donner aux membres du personnel en contact avec les clients des informations détaillées sur les informations personnelles des clients, l'historique des achats, les préférences et les préoccupations d'achat.


\section{Motivations}    

KYC est un moyen de rendre la vérification de l'identité des clients plus précise et moins vulnérable à la fraude.

KYC doivent être effectuées lors de l'intégration d'un nouveau client, mais il est préférable de répéter ces vérifications de temps en temps, pour s'assurer que tout est comme il se doit. En surveillant les comptes clients de cette manière, les comportements suspects peuvent être signalés plus rapidement.

Un système CRM fournit des flux de travail automatisés qui permettent à votre équipe marketing de consacrer plus de temps à des tâches stratégiques, telles que la création de campagnes marketing qui résonnent, l'analyse des données de ces campagnes et le test de différentes approches basées sur ces analyses. Les agents du service client peuvent passer leur temps à travailler avec des clients qui ont des questions, des problèmes ou des besoins plus complexes. En bref, avec des processus de service client plus efficaces, les entreprises peuvent établir de meilleures relations avec leurs clients.

\section{Problématiques}	

En réalité, la réalisation d'une application,qui applique le principe de KYC et integre un  système CRM,
nécessite
de faire face à des problématiques diverses et complexes. Ainsi, la société a décidé de se contenter,
dans un premier temps, Mise en place d’un système d’extration des donnees à partir des images (carte d'identité ou passeport) et traitement des ces donnees.
Ce sujet soulève de nombreuses questions aux implications différentes. Comment peut extraire le texte apartir de l'image? Comment sera-t-il traité ? Comment peut-il être utilisé dans le principe KYC ? Comment pouvons-nous obtenir un système CRM intégré ?


\section{Objectifs}

La mise en place d'une application pour appliquer l'ide de KYC en basant sur les différent technologie disponible . En basan sur l'extraction du text apartir d'une imange OCR on peut extracter la code MRZ apartir d'une imange du piece d'idendite ou passport est passe le code a un algorithem qui permer de d'etecter les information personnel.



		
	\chapter{Présentation de la société}
\label{chap:introduction}
%\pagenumbering{arabic}
\section{Introduction}
\begin{figure}[h]
	\includegraphics[scale=0.14]{/home/mohamed/mes_fichier/CADORIM/LaTeX/pfe/Template LaTeX/Images/cado_logo.png}
	\centering
	\caption{CADOROM}
\end{figure}
CADORIM est une société de transfert d’argent mauritanienne basée à Nouakchott,
fondée fin 2018 par un entrepreneur mauritanien, titulaire d'un doctorat en
mathématiques,
CADORIM consiste a transférer de l’argent depuis n’importe quel pays dans le
monde vers ses proches en Mauritanie. Notre objectif et de fourni une plateforme
numérique permet à l’utilisateur de régler ses commandes en toute sécurité et
confidentialité assurée par le service de PayPal qui est mondialement connu pour sa
fiabilité et simplicité.t Pour effectuer un paiement il suffit d'une simple carte bancaire
ou un compte PayPal . et une éventuelle possibilité de virement bancaire.
CADORIM a été élu comme le champion de Banque Centrale de Mauritanie (BCM )
1ère édition 2019 Fintech Challenge,
Le siège social de CADORIM est situé à marche capital , Nouakchott, Mauritanie,
immatriculée au registre du commerce.
\section{Missions}
CADORIM offre une large palette de prestations organisées autour des activités suivantes :
\begin{enumerate}
	
	\item Maintenance et amélioration de leurs propres applications (CadoRim et MauriPay)
	\item Développement des applications 
	\item Des agences des reçoivent d'argent et de service client
	
\end{enumerate}
	\chapter{Environnement de développement}
\label{sec:EnvironnementDeTravail}
Durant la réalisation de ce projet, nous avons essayé d’utiliser différents
outils de développement, d’une part afin de rendre la tâche de la
réalisation plus facile, d’autre part pour que notre système soit robuste et
répond parfaitement a nos besoins , et que nos interfaces soient claires et
faciles à utiliser.
\section{Gestion du projet}
Cette section est pour la présentation de mé-stratégie de fonctionnement et le programme de gestion de projets que j'ai utilisé pour lancer, structurer et gérer le travail sur ma solution.
\subsection{Méthodologie de travail}
C'est la façon de conduire un processus de développement. Il s’agit d'un procédé un assemblage d’étapes ou procédures à mettre en œuvre dans un raisonnement méthodologique, accompagné d’outils et de techniques. L' application d'une méthode est incontournable dans l'entreprise de tout projet, spécialement dans la réalisation de projets informatiques. \newline
L'obligation de l'utilisation de ces méthodes trouve sa justification dans un certain nombre de facteurs :
\begin{enumerate}
	\item Plusieurs projets informatiques dans les passes sont échoués gras à un manque d'organisation ou la non-satisfaction des besoins;
	\item La révolte de l'industrie des applications provoquée par des défaillances logicielles qui ont introduit de nouveaux éléments pour assurer la qualité des applications : le génie logiciel ;
	\item Diverses exigences liées au coût, au temps et aux complexités des logiciels informatiques.
\end{enumerate}

L'application de procédé de développement de logiciels permet ainsi la préparation de systèmes informatiques de manière crédible et faisable tout en répondant à l'ensemble des obligations du client et du génie logiciel.
\newline \newline
Il existe diverses méthodes de développement informatique. J'illustre deux approches : l’approche traditionnelle et l’approche agile. Les deux approches se distinguent  dans la façon de diviser le projet. Les méthodes  rationnelles ou opérationnelles ou encore traditionnelles se sont imposé les premier.
\subsubsection{L'approche traditionnelle}
Cette approche s’aviser directement de l’architecture des ordinateurs. Les méthodes traditionnelles prêchent une démarche purement planifiée avec un enchaînement d'activités bien définies. La suite des activités et le plan doivent être clairement respectés et aucun changement n’est permis. Il est prévu du client une norme des besoins globaux, détaillée, claire, précise et validée en entrée. Ainsi, tout doit être prévisible, du début du projet à la livraison du produit, d’où l’appellation de méthodes prédictives. \newline
D'après le plan approuvé les méthodes cartésiennes proposent plusieurs modèles d’exécution des activités du projet :
\begin{enumerate}
	\item Le \textbf{modèle en cascade} : Sur ce modèle, la procédure de développement est découpée séquentiellement et de façon linéaire selon les activités intégrales du cycle de vie du développement logiciel : l’analyse, la conception, le codage et les tests. Le plan de déavancement des phases (planification prédictive) est établi en tout début de processus. La transition à une phase donnée n’est fait que si le résultat de la phase précédente a été validé et estimé satisfaisant par le client et les utilisateurs.
	\item Le \textbf{modèle en V} : Le cycle en V est à la base de tout développement logiciel, il en représente les activités intrinsèques. Il tient d'avantage compte de la réalité que le modèle en cascade, le processus de développement n’est pas réduit à un enchainement de tâches séquentielles. Le modèle en V permet d’anticiper sur les phases ultérieures de développement du produit en particulier les plans de test de qualifications et de performance.
\end{enumerate}
Parmi les méthodes traditionnelles, nous pouvons citer : SADT, CORIG, …
\subsubsection{L'approche agile}
Cette approche est définie par les concepts suivants : la simplicité, la légèreté, la souplesse, un lien fort avec le client. C’est dans cette optique que certains apparentent le développement agile aux notions de flexibilité, de rétroaction et d’adaptation au changement rapide et continu.
\newline
Une méthode agile est une approche itérative et incrémentale, qui est menée dans un esprit collaboratif avec juste ce qu’il faut de formalisme. Elle génère un produit de haute qualité tout en prenant en compte l’évolution des besoins des clients et en anticipant sur les risques. Il y’a continuellement des aller et retour avec le client. L’application logicielle est livrée par version incrémentale. Les versions successives sont aussi fiables que le livrable final en termes de tests et de validation. En quelque sorte le processus est déroulé comme un enchaînement de « mini-cascades ». A chaque nouvelle itération, l’ensemble de l’architecture et de la conception logicielle est reconsidéré, le code est retravaillé.
\newline
Les méthodes agiles aspirent donc à améliorer la réactivité et l’adaptabilité des sociétés de logiciels et constituent un moyen de survie dans un environnement instable en s’accompagnant des valeurs suivantes :
\begin{enumerate}
	\item Les individus et les interactions plutôt que les processus et les outils;
	\item L’application fonctionnelle plutôt que la documentation compréhensive;
	\item La collaboration avec le client plutôt que la négociation des contrats;
	\item La réponse au changement plutôt que le suivi d’un plan.
	\newline
\end{enumerate}
L’agilité comprend plusieurs courants de pensée qui ont conduits à des méthodes différentes, reposant sur les mêmes concepts mais présentant des singularités. Les méthodes Scrum, Kanban, et XP (eXtreme Programming) sont des exemples de ces méthodes.
\newline\newline

\textbf{La méthode SCRUM}
\newline
\textbf{Scrum} est une méthode agile de gestion de projet qui permet de produire la plus grande valeur métier dans la durée la plus courte. Elle a pour objectif d’améliorer la cohésion de l’équipe et la rapidité du processus de développement. Le nom Scrum renvoie à une pratique généralement connue au rugby signifiant la « mêlée ». \newline
Cette méthode qualifie un ensemble de rôles, d’instruments de gestion et de pratiques managériales favorisant un environnement basé sur la transparence, l’inspection, le suivi et l’adaptation. Le cycle de vie d’un projet Scrum peut être découpé en trois parties :
\begin{enumerate}
	\item Phase d'\textbf{initiation ou démarrage} : il s’agit d’une phase linéaire où l’on définit le périmètre fonctionnel du système et la liste des fonctionnalités (\textbf{Backlog}) agencées par ordre de priorité, d’effort, de complexité et de risque. C’est aussi à ce niveau que l’architecture est définie.
	
	\item Phase de \textbf{développement} est un processus empirique : le projet est découpé en cycles itératifs d’une durée de deux semaines ou \textbf{sprints}. Chaque sprint regroupe une ou plusieurs fonctionnalités du Backlog. Tout au long de cette phase, le travail réalisé est mesuré et contrôlé et une amélioration constante du prototype est faite.
	
	\item Phase de \textbf{Clôtures} est une phase linéaire de gestion de la livraison du produit final.
\end{enumerate}
La figure \ref{Scrum} montre l’articulation générale de Scrum. \newline
\begin{figure}[h]
	\includegraphics[width=15cm, height=5cm]{./Template LaTeX/Images/scrum1.jpeg}
	\centering
	\caption{Articulation générale de la méthode Scrum}
	\label{Scrum}
\end{figure}
Les responsabilités managériales sont réparties sur trois rôles fondamentaux :
\begin{enumerate}
	\item \textbf{Scrum Master}
	\item \textbf{Product Owner}
	\item \textbf{Équipe Scrum}
	\newline
\end{enumerate}
Les artéfacts et pratiques de Scrum
\begin{enumerate}
	\item \textbf{Product Backlog} : état courant des tâches à accomplir;
	\item \textbf{Sprint} : itération de deux semaines;
	\item \textbf{Effort-Estimation} : permanente sur les entrées du Backlog;
	\item \textbf{Sprint Backlog} : Product Backlog limité au sprint en cours;
	\item \textbf{Daily Scrum Meeting} : ce qui a été fait, ce qui reste à faire;
	\item \textbf{Sprint Review Meeting} : Présentation des résultats du Sprint. \ref{Scrum}
	\newline \newline \newline
\end{enumerate}
\textbf{SCRUM contre KANBAN}
\newline
Scrum est plus prescriptif que Kanban, qui évite de définir les rôles et les équipes et qui n’a pas de structure formelle de réunions. Kanban ne prescrit pas non plus d’itérations – bien qu’elles puissent être incorporées si vous le souhaitez. \newline
Les techniques de visualisation des processus de Kanban le rendent idéal pour les équipes colocalisées qui travaillent sur un backlog d’éléments sujets à des changements fréquents (par exemple, Kanban est souvent utilisé par les équipes de support). \newline
Le tableau Kanban est cependant souvent adopté par les équipes Scrum sous la forme d’un tableau de tâches et est utilisé pour suivre la progression tout au long d’un sprint. \newline
La limite de la règle Work In Progress dans Kanban la rend également adaptée aux équipes ayant des ressources limitées ou lorsque l’entrée de chaque membre est requise sur chaque élément. Cela pourrait s’appliquer, par exemple, à une équipe de communication au sein d’une grande organisation. \newline
Alors que Scrum limite la quantité de travail dans chaque sprint, la charge de travail est déterminée par l’estimation relative de la taille de chaque histoire (en points) et est approuvée par l’équipe Scrum à chaque session de planification. \newline
Alors qu’une équipe Kanban suit le « temps de cycle » et optimise les délais d’exécution aussi courts et prévisibles que possible, une équipe Scrum vise à améliorer son rendement sur les sprints successifs et à améliorer la « vélocité » de l’équipe (le nombre de points d’estimation relatifs complétés dans un sprint). Cela rend sans doute Scrum plus adapté à la mise à l’échelle – il semble certainement plus familier et prévisible, ce qui peut être rassurant pour les grandes organisations. \newline\newline
\textbf{SCRUM contre XP}
\newline
Dans Scrum, les équipes et les réunions sont assez gravées dans le marbre \footnote{Dans l’antiquité, les engagements pour la constructions de bâtiments importants étaient gravés sur des plaques de marbre (Athènes : arsenal du Pirée, Delphes). Les travaux s’étendant sur de nombreuses années, on ne pouvait faire confiance aux tablettes de cire ou aux papyrus. Sur ces plaques, on définissait par exemple la grandeur du bâtiment ou le montant des amendes pour les retards. Ce qui n’était pas « gravé dans le marbre » n’était donc pas contractuel. Voir le lien \href{https://fr.wiktionary.org/wiki/graver_dans_le_marbre}{https://fr.wiktionary.org/wiki/graver-dans-le-marbre}} alors que la question de savoir comment le travail est réellement fait est laissée aux équipes pour décider elles-mêmes. XP, d’autre part, est livré avec un ensemble de pratiques de base qui pourraient sembler accablantes pour le débutant Agile. \newline
On pourrait dire que Scrum est une méthodologie, qui est plus concernée par la productivité tandis que XP est plus préoccupé par l’ingénierie. \newline
La valeur que les pratiques XP peuvent ajouter est incontestable et de nombreuses organisations qui utilisent Scrum adoptent la programmation par paires, le développement piloté par les tests et le refactoring comme des pratiques qui améliorent la qualité, accélèrent le processus de publication et / ou réduisent le besoin de revoir le travail en raison de la dette technique. \newline
Outre les itérations plus courtes, d’autres éléments importants qui différencient XP de Scrum sont les suivants :
\begin{enumerate}
	\item Les équipes XP travaillent sur des éléments dans un ordre de priorité strict alors qu’une équipe Scrum ne s’attaque pas nécessairement à chaque élément dans l’ordre de priorité une fois dans le sprint;
	\item Les équipes XP peuvent intégrer de nouveaux éléments de travail dans une itération et changer d’éléments de taille équivalente (tant qu’ils n’ont pas été démarrés) si le client décide d’une nouvelle priorité.
	
	%%%%%%%%%%%%%%%%%%%%%%%%%%%%%%%%%%%%%%%%%%%%%%%%%%%%%%%%%%%%%%%%%%%%%%%%%%%%%%%%%%%%%%%%%%%%%%
\end{enumerate}
En termes de similitudes, le rôle du client dans XP est très similaire à celui du Product Owner dans Scrum – en ce sens qu’ils aident à écrire des user stories, à les hiérarchiser et sont toujours disponibles pour les développeurs – bien que moins bien définis. \newline
Scrum et XP imposent tous deux une réunion debout quotidienne. Bien que les deux soulignent l’importance de la co-localisation, seul XP le rend décisif. Voir le site \href{https://manifesto.co.uk/kanban-vs-scrum-vs-xp-an-agile-comparison/}{https://manifesto.co.uk/kanban-vs-scrum-vs-xp-an-agile-comparison/}.


\subsection{Logiciel de gestion du projet : Trello}
Trello est un outil de gestion de projet en ligne, lancé en septembre 2011 et inspiré par la méthode Kanban. Il repose sur une organisation des projets en planches listant des cartes, chacune représentant des tâches. Les cartes sont assignables à des utilisateurs et sont mobiles d'une planche à l'autre, traduisant leur avancement. Pour en savoir plus, veillez visiter le lien \href{https://fr.wikipedia.org/wiki/Trello}{https://fr.wikipedia.org/wiki/Trello}.

\section{Conception}
Dans cette section, je présente le langage et le logiciel de modélisation que j'ai utilisé pour concevoir notre solution.

\subsection{Langage de modélisation : UML}
UML est un langage de modélisation orientée objet permettant aux développeurs de modéliser un système d’information en considérant plusieurs vues chacune reflétant un aspect comportemental \footnote{Diagramme de cas d’utilisation, diagramme d’activité, diagramme de séquence, diagramme d’interaction, etc.} ou structurel \footnote{Diagramme de classe, diagramme de composants, diagramme de déploiement, diagramme de structure composite, etc.} du système.
\newline
En effet, nous avons opté pour UML au détriment de la MERISE car nous avons besoin
d’une approche de conception prenant en considération l’aspect orienté objet pour :
\begin{enumerate}
	\item Pouvoir mettre le focus sur le rôle temporel des instances d’objets lors de déclenchement desactions (à travers le diagramme de séquence) ;
	\item Faciliter par la suite la génération des classes DAO à partir du diagramme de classes.
\end{enumerate}
Certes, UML est très riche en matière de modélisation et propose au total 13 diagrammes chacun fournissant une vision particulière du système à concevoir. Dans notre contexte, je me suis limités à 4 diagrammes explicités sur le tableau \ref{3.1}.

\subsection{Logiciel de modélisation}
Modelio est un logiciel open source et multiplateforme permettant, entre autres, la modélisation UML et Business Process Model and Notation (BPMN). Pour en savoir plus, veuillez visiter le lien \href{https://www.modelio.org/about-modelio/features.html}{https://www.modelio.org/about-modelio/features.html}.
\newline
Sans doute, les logiciels de modélisation UML sont nombreux, à savoir, Visual Paradigm, Eclipse Papyrus, StarUML, PowerDesigner, Umbrello, etc. Vu que les diagrammes UML que nous voulons réaliser sont disponibles dans tous ces logiciels, il n’y avait pas en effet un choix à argumenter car
tous les choix étaient satisfaisants. Mais, de façon subjective, nous pouvons préciser que l’avantage de Modelio dans notre contexte est le fait que je m'y suis	 déjà habitués. Les diagrammes que nous avons réalisés avec Modelio sont ceux mentionnés ci-après.
\newline\newline
\begin{table}[h]
	\begin{tabular}{|m{6cm}|m{10cm}|}
		\hline
		\textbf{Diagramme} & \textbf{Rôle} \\
		\hline
		Diagramme de cas d’utilisation & Présenter les acteurs du système, ses fonctionnalités, les relations entre les acteurs et entre les fonctionnalités. \\
		\hline
		Diagramme d’activité & Déterminer l’enchaînement des différentes étapes qui composent une fonctionnalité du système. \\
		\hline
		Diagramme de séquence & Fournir une vue détaillée du diagramme d’activité en mettant le focus sur l’ordre chronologique et sur les objets crées et les méthodes appelées. \\
		\hline
		Diagramme de classe & Représenter la structure interne du système sous forme de classes et d’interfaces et préciser les différentes relations entre elles. \\
		\hline
		
	\end{tabular}
	\caption{Rôles des diagrammes UML utilisés.}
	\label{3.1}
\end{table}

\section{Implémentation}
Dans cette section, je présente les langages, les logiciels, les frameworks et les motifs d’architecture que j'ai utilisé.

\subsection{Front-end}
\subsubsection{Editeur de texte : VS Code}
VS Code (Visual Studio Code) est un éditeur de code source réalisé par Microsoft pour Windows, Linux et macOS. [9] Les fonctionnalités incluent la prise en charge du débogage, lacoloration syntaxique, la saisie semi-automatique intelligente du code, les extraits de code, la refactorisation du code et Gitintégré. Les utilisateurs peuvent modifier le thème,les raccourcis clavier,les préférences et installer des extensions qui ajoutent des fonctionnalités supplémentaires. Pour en savoir plus, veillez visiter le lien \href{https://en.wikipedia.org/wiki/Visual_Studio_Code}{https://en.wikipedia.org/wiki/VisualStudioCode}.
\subsubsection{Languages}
\begin{table}[h]
	\begin{tabular}{|m{6cm}|m{10cm}|}
		\hline
		\textbf{Langage} & \textbf{Contexte d’utilisation} \\
		\hline
		Dart & Création des applications moible\\
		\hline
		PHP  & Langage de programmation libre, principalement utilisé pour produire des pages Web dynamiques via un serveur HTTP\\
		\hline
		
	\end{tabular}
	\caption{Contexte d’utilisation des différents langages utilisés.}
\end{table}
\subsubsection{Framework : Flutter}
\label{Flutter}
Flutter est un cadre de développement d’applications mobiles open source permettant de développer des applications mobiles natives Andriod et iOS en un seul code.

Flutter a été introduit par Google. La version stable de Flutter est Flutter 1.0 qui a été publiée le 4 décembre 2018. Le ciel est la première application Flutter qui a fonctionné dans l’OS Andriod. \href{https://www.claudebueno.com/technologies/introduction-a-flutter.htm}.
\subsubsection{Framework Web :  Laravel}
\label{Laravel}
Laravel est un framework d'application Web avec une syntaxe expressive et élégante. Nous croyons que le développement doit être une expérience agréable et créative pour être vraiment épanouissante. Laravel tente de simplifier le développement en facilitant les tâches courantes utilisées dans la majorité des projets Web, telles que l'authentification, le routage, les sessions et la mise en cache.

Laravel vise à rendre le processus de développement agréable pour le développeur sans sacrifier les fonctionnalités de l'application. Les développeurs heureux font le meilleur code. À cette fin, nous avons tenté de combiner le meilleur de ce que nous avons vu dans d'autres frameworks Web, y compris des frameworks implémentés dans d'autres langages, tels que Ruby on Rails, ASP.NET MVC et Sinatra.

Laravel est accessible, mais puissant, fournissant des outils puissants nécessaires pour les applications volumineuses et robustes. Une superbe inversion du conteneur de contrôle, un système de migration expressif et une prise en charge des tests unitaires étroitement intégrée vous offrent les outils dont vous avez besoin pour créer n'importe quelle application qui vous est confiée.
\href{https://laravel.com/docs/4.2/introduction#laravel-philosophy}.
\subsubsection{IDE : Android Studio}
Android Studio est l’IDE officiel pour le système d’exploitation Android de Google,construit sur le logiciel IntelliJ IDEA de JetBrainset conçu spécifiquement pour le développement Android. Pour en savoir plus veillez visiter le lien \newline \href{https://en.wikipedia.org/wiki/Android_Studio}{https://en.wikipedia.org/wiki/AndroidStudio}. \newline
Essentiellement, je l'ai utilisé pour lancer l'application sur android.
\subsection{Back-End}
\subsubsection{IDE : Visual Studio}
Microsoft Visual Studio est un IDE de Microsoft. Il est utilisé pour développer des programmes informatiques, ainsi que des sites Web, des applications Web, des services Web et des applications mobiles. Pour en savoir plus, veillez visiter le lien \href{https://en.wikipedia.org/wiki/Microsoft_Visual_Studio}{https://en.wikipedia.org/wiki/MicrosoftVisualStudio}.
\subsubsection{MVC}
\label{3.3.3.1}
L'objectif du motif MVC (Model-View-Controller ou Modèle-Vue-Contrôleur) est un modèle dans la conception de logiciels. Il met l'accent sur la séparation entre la logique métier et l'affichage du logiciel. Cette «séparation des préoccupations» permet une meilleure répartition du travail et une maintenance améliorée. Le tableau ci-dessous en présente une explication.
\newline\newline
\begin{table}[h]
	\begin{tabular}{|m{6cm}|m{10cm}|}
		\hline
		\textbf{Couche logicielle} & \textbf{Rôle} \\
		\hline
		Modèle & Gère les données et la logique métier \\
		\hline
		Vue & Gère la disposition et l'affichage \\
		\hline
		contrôleur & Chemine les commandes des parties "model" et "view" \\
		\hline
	\end{tabular}
	\caption{Rôle des trois couches logicielles du motif MVC.}
\end{table}
\newline La figure ci-dessous explique les moments d'intervention de chaque couche du motif MVC.
\begin{figure}[h]
	\includegraphics[width=15cm, height=8cm]{./Template LaTeX/Images/mvc2.png}
	\centering
	\caption{Rôle des trois couches logicielles du motif MVC}
\end{figure}
\section{Sécurité}
Dans cette section, je présente les algorithmes et techniques de chiffrement que nous avons
utilisés pour sécuriser davantage l'application.
\subsection{Hachage des mots de passe}
	\chapter{Analyse fonctionnelle et conceptuelle}
\label{sec:AnalyseFoncEtConcep}

Dans ce chapitre, je passe à la phase de conception de l'application dans lequel j'explique en détails les différents diagrammes UML relatifs a l'analyse fonctionnelle.

\section{Analyse fonctionnelle}
L'analyse fonctionnelle peut être expliqué comme une traduction du cahier de charges en une langage de conception (UML dans notre cas) permettant de caractériser les fonctionnalités d'un logiciel de façon plus compréhensible, répartissable, et réalisable. Pour ce faire, je me suis posés, essentiellement, trois questions, à savoir :
\begin{enumerate}
	\item \textbf{Qui peut faire quoi ?} Pour répondre à cette question, je dois décrire non seulement l’ensemble des acteurs (\textbf{qui}) et l'ensemble des fonctionnalités du système (\textbf{quoi}) mais aussi les privilèges de chaque acteur (\textbf{peut faire}), les relations entre acteurs et entre fonctionnalités ;
	
	\item \textbf{Comment ?} La question étant « Comment un acteur procède à une fonctionnalité ? », il s’agit donc de détailler comment se déroule le processus d’interaction entre l’acteur et le système lorsque ce premier sollicite une fonctionnalité de ce dernier ;
	
	\item \textbf{Quand ?} Cette question reprend la question précédente mais s’intéresse plutôt à l’aspect temps : à quel moment se déroule chaque étape du processus d’interaction ? Comment les étapes se succèdent dans l’ordre chronologique ?
	
\end{enumerate}

Le diagramme de cas d’utilisation permet de répondre à la première question, le diagramme d’activité répond à la seconde question et le diagramme de séquence à la troisième.

\subsection{Diagramme de cas d’utilisation}
Dans cette section, j'identifie le système, les acteurs et les cas d'utilisations.

\subsubsection{Système}
\label{4.1.1.1}
Le système représente une application mobile, nommée \textbf{Ri3aya} permettant aux patient de réserver des rendez-vous a domicile avec des consultants médicales (médecins, infirmiers, pharmacien, etc). \newline
L’application \textbf{Ri3aya} est composée de deux espaces, à savoir, l’espace Consultant Médical et l’espace Patient.
Les espaces que comprend cette application sont décrits à travers le tableau \ref{4.1} :
\begin{table}[h]
	\begin{tabular}{|m{6cm}|m{10cm}|}
		\hline
		\textbf{Espace} & \textbf{Fonctionnalités} \\
		\hline
		Patient & -	Créer un compte en tant que Patient \newline
		- Réinitialiser le mot de passe \newline
		- Modifier les informations du profil \newline
		- Rechercher une consultation \newline
		- Demander un rendez-vous \newline
		- Payer les frais d’un rendez-vous \newline
		- Retirer une demande avant être accepter \newline
		- Visualiser la liste de ses rendez-vous \newline
		- Visualiser les détails d’un rendez-vous \newline
		- Visualiser le bilan statistique \newline
		- Visualiser les notifications reçues \\
		\hline
		Consultant Médical & - Créer un compte en tant que Consultant Médical \newline
		- Réinitialiser le mot de passe \newline
		- Rectifier les informations du profil \newline
		- Visualiser la liste des rendez-vous \newline
		- Visualiser les informations d’un patient \newline
		- Accepter un rendez-vous \newline
		- Refuser un rendez-vous \newline
		- Planifier, initialement, les horaires de consultation \newline
		- Mettre à jour les horaires de consultations \newline
		- Reporter un rendez-vous pour un patient \newline
		- Visualiser le bilan statistique \newline
		- Visualiser les notifications reçues \\
		\hline
	\end{tabular}
	\caption{Modules de l’application.}
	\label{4.1}
\end{table}

\subsubsection{Acteurs du système}
Les fonctionnalités du système peuvent être sollicitées par deux types d’acteurs, à savoir :
\begin{enumerate}
	\item \textbf{Patient :} c'est toute personne demandant un rendez-vous d'un consultant médical;
	\item \textbf{Consultant Médical :} c’est l'acteur que les patients demandent des rendez-vous. J'ai lui désigné le terme « Consultant Médical » puisqu'il englobe le médecin, l'infirmier, le pharmacien et tout personne dans le domaine médical pouvant être sollicité.
\end{enumerate}

\subsubsection{Cas d'utilisation}
Les figures \ref{Figure 4.1}, \ref{Figure 4.2} et \ref{Figure 4.3} présentent les diagrammes de cas d’utilisation des acteurs du système.
\begin{figure}[h]
\includegraphics[scale=0.3]{D:/NTFS 3/Mon_master/M2/S4/Diagrammes/Capt}
	\centering
	\caption{Diagramme de cas d'utilisation d'utilisateur}
	\label{Figure 4.1}
\end{figure}
\begin{figure}[h]
\includegraphics[scale=0.3]{D:/NTFS 3/Mon_master/M2/S4/Diagrammes/Capt2}
	\centering
	\caption{Diagramme de cas d'utilisation du patient}
	\label{Figure 4.2}
\end{figure}
\begin{figure}[h]
\includegraphics[scale=0.3]{D:/NTFS 3/Mon_master/M2/S4/Diagrammes/Capt3}
	\centering
	\caption{Diagramme de cas d'utilisation du consultant médical}
	\label{Figure 4.3}
\end{figure}

\subsection{Diagrammes d’activité}
Cette section a pour objectif de mettre en surbrillance le processus quelques fonctionnalités de l'application pour voir les détailles. J'ai choisi trois cas d’utilisation, à savoir, la création d’un compte, la demande d’un rendez-vous et l’ajournement d’un rendez-vous.

\subsubsection{Création de compte}
Le processus de création de compte commence tout d’abord par la saisie d’informations requises. Ensuite, l’utilisateur soumet une demande de création du compte. Pour des raisons de sécurité, afin de valider la création de compte, nous envoyons au numéro de téléphone de l’utilisateur un SMS comprenant un code \gls{OTP} qui serait expiré dans 2 minutes. Cela nous permettrait de vérifier qu’il est bien le sien. L’envoi des SMS est effectué via l’Application Programming Interface (\gls{API}) Firebase Authentication. L’application bloquerait l’utilisateur pour une durée de quatre heures après avoir reçu 5 codes \gls{OTP} sans utiliser aucun. Si l’utilisateur reçoit le SMS, Firebase à travers Google Play Services essaierait de récupérer automatiquement le code \gls{OTP} du SMS; si l’opération réussit, un courrier de bienvenue serait envoyé à l’utilisateur et la création du compte s’achève avec succès. Toutefois, si Firebase n’arrive pas à récupérer le SMS reçu, l’application demande à l’utilisateur de saisir manuellement le code \gls{OTP}. Voir la figure \ref{Figure 4.4}.
\begin{figure}[h]
	\includegraphics[scale=0.3]{D:/NTFS 3/Mon_master/M2/S4/Diagrammes/Capture1}
	\centering
	\caption{Diagramme d'activité : Création de compte}
	\label{Figure 4.4}
\end{figure}

\subsubsection{Demande de rendez-vous}
Pour une demande de rendez-vous, il faut d'abord rechercher une consultation par consultant médical ou par spécialité. Une fois le patient choisit le profil du consultant médical qui lui satisfait, il peut directement lancer la demande de rendez-vous. Une notification sera envoyée au consultant médical en question, ensuite, le patient aussi recevra une notification de la décision du consultant médical concernant le rendez-vous. S'il s'agit d'une approbation, le patient doit procéder au paiement des frais de consultation qui doit être effectue en Bankily. Par contre s'il s'agit un rejet, le processus terminera par la notification. Voir la figure \ref{Figure 4.5}.
\begin{figure}[h]
	\includegraphics[scale=0.35]{D:/NTFS 3/Mon_master/M2/S4/Diagrammes/Capture2}
	\centering
	\caption{Diagramme d'activité : Demande de rendez-vous}
	\label{Figure 4.5}
\end{figure}

\subsubsection{Ajournement de rendez-vous}
Le patient peut lancer une demander de report de rendez-vous, dans ce cas une notification sera envoyée au consultant médical en question. Ce dernier devra proposer au patient la date et l'horaire du nouveau rendez-vous. Si le patient confirme la proposition, une notification à ce propos sera envoyée au consultant médical concerne. En revanche, si les nouvelles date et horaire ne satisfait pas au patient, il pourrait les rejeter, avec la possibilité d’indiquer sous forme de commentaire les dates et les horaires de préférence, et demander au consultant médical de proposer une nouvelle date et horaire. Voir la figure \ref{Figure 4.6}.
\begin{figure}[!h]
	\includegraphics[scale=0.4]{D:/NTFS 3/Mon_master/M2/S4/Diagrammes/Capture3}
	\centering
	\caption{Diagramme d'activité : Ajournement de rendez-vous}
	\label{Figure 4.6}
\end{figure}

\subsection{Diagrammes de séquence}
L'objectif de cette section est de mettre l'accent sur l’aspect temps du déroulement de processus des fonctionnalités pour lesquelles nous avons fait les diagrammes d’activité. Voir les figures \ref{Figure 4.7} et \ref{Figure 4.8}.
\begin{figure}[h]
	\includegraphics[scale=0.3]{D:/NTFS 3/Mon_master/M2/S4/Diagrammes/Diagramme de séquence (Signup)}
	\centering
	\caption{Diagramme de séquence : Création de compte}
	\label{Figure 4.7}
\end{figure}
\begin{figure}[h]
	\includegraphics[scale=0.3]{D:/NTFS 3/Mon_master/M2/S4/Diagrammes/Diagramme de séquence (Demande de rendez-vous)}
	\centering
	\caption{Diagramme de séquence : Demande de rendez-vous}
	\label{Figure 4.8}
\end{figure}

\section{Modélisation de la base de données}
\subsection{Diagramme de classes}
Suite à l'analyse fonctionnelle de l'application, je m'oriente désormais la modélisation de la base de données. C’est dans ce cadre que j'ai réalisé le diagramme de classes qui permet d’identifier les différentes entités \footnote{il s’agit des classes qui seront converties en tables.} du système et les relations entre elles. \newline
Les points suivants résument les principales contraintes sur lesquelles je me suis basé pour réaliser le diagramme de classes :
\begin{enumerate}
	\item L'entité \textbf{User} désigne la classe qui englobe tout type d'utilisateur : Patient et Consultant Médical;
	\item L'entité \textbf{Consultation} C'est la classe par laquelle l'utilisateur initialise ses horaires disponibles pour les rendez-vous;
	\item La fonction du consultant médical représente une entité nommée \textbf{Function};
	\item Le lieu de travail du consultant médical représente aussi une entité sous le nom \textbf{Workspace};
\end{enumerate}
La figure \ref{Figure 4.9} représente le diagramme de classes.
\begin{figure}[h]
	\includegraphics[scale=0.6]{D:/NTFS 3/Mon_master/M2/S4/Diagrammes/Diagramme de classe}
	\centering
	\caption{Diagramme de classes.}
	\label{Figure 4.9}
\end{figure}
	\let\cleardoublepage\clearpage

\chapter{Implémentation}
\label{sec:implementation}

Dans ce chapitre, j'essaie de faire une évocation globale des détails techniques relatifs au développement de l’application \textbf{Ri3aya}. Pour ce faire, je présente, tout d’abord, l’architecture logicielle de l'application. Ensuite, j'illustre les services web que j'ai réalisé pour pouvoir interagir entre le \gls{SGBD} et l’application. En outre, je décris les différents motifs d’architecture que j'ai implémenté. Après cela, j'explose comment j'ai procédé à l'obfuscation du code source pour le protéger contre les attaques de rétro-ingénierie. Enfin, à travers des captures d’écran, je réalise une démonstration des principales fonctionnalités de l’application.

\section{Architecture de l’application}
L’architecture de notre application peut être divisée en deux blocs, le premier concerne le côté front-end (IONIC), le second est relatif au côté back-end (services web et \gls{SGBD}). La communication entre ces deux blocs est garantie à travers l’architecture Service-Oriented Architecture (\gls{SOA}). La figure ci-dessous en présente une illustration.
\begin{figure}[h]
	\includegraphics[scale=0.79]{D:/NTFS 3/Mon_master/M2/S4/Diagrammes/Architecture}
	\centering
	\caption{Architecture logicielle de l'application.}
	\label{Figure 5.1}
\end{figure}
\newline
Le bloc à gauche constitue le côté front-end de l'application. J'ai utilise \textbf{IONIC} (voir la section \ref{IONIC}) avec \textbf{ANGULAR} (voir la section \ref{ANGULAR}), ce dernier repose essentiellement sur une approche \gls{MVC} que nous avons déjà expliqué dans la section \ref{3.3.3.1}. \newline
Le bloc à droite représente le coté back-end de l'application. Autrement dit, il s'agit du serveur de l'application, là ou je récupère les requêtes \gls{HTTP} provenant du front-end, je les traite et je les fait retourner au front-end. \newline
Toutes les requêtes de base de données sont gérées par un RESTful \gls{API} regroupant un ensemble de services web réalisés via l'\gls{ORM} Entity Framework dans le \textbf{Data Layer} qui représente le couche \gls{DAO} (voir la section \ref{DAO}). Ainsi, lorsque l’application aurait besoin d’exécuter une requête, elle communique avec le Contrôleur correspondant, à travers un tunnel \gls{HTTP}, qui lui retourne le résultat sous format JavaScript Object Notation (\gls{JSON}) comme la figure ci-dessus montre.

\section{Services web}
Je présente dans cette section les essentiels scripts C\# dans le \textbf{Data Layer} qui constituent les services web du back-end. \newline
Le tableau ci-dessous en présente la liste.
\begin{table}[h]
	\begin{tabular}{|m{6cm}|m{10cm}|}
		\hline
		\textbf{Service web} & \textbf{Rôle} \\
		\hline
		AuthRepository.cs & Authentification d'utilisateurs \\
		\hline
		AdressProvider.cs & Insertion, récupération, modification et suppression des données des adresses \\
		\hline
		ConsultationProvider.cs & Insertion, récupération, modification et suppression des données des consultations \\
		\hline
		FunctionProvider.cs & Insertion, récupération, modification et suppression des données des fonctions des consultant médical \\
		\hline
		UserProvider.cs & Insertion, récupération, modification et suppression des données des utilisateurs \\
		\hline
		WorkspaceProvider.cs & Insertion, récupération, modification et suppression des données des lieux de travails des consultants médical \\
		\hline 
		UserWorkspaceProvider.cs & Insertion, récupération, modification et suppression des données de la table associative entre "User" et "Workspace" \\
		\hline
	\end{tabular}
	\caption{Rôles des différents services web de l'application.}
\end{table}

\section{Vue sur l’application}
Dans cette section, je présente les principales fonctionnalités dans chaque espace de l'application.
\subsection{Options d’internaute}
Un internaute désigne tout individu qui navigue dans l’application sans qu’il soit connecté à un compte. \newline 
La première figure \ref{Figure 5.2} présente l’écran d’accueil de l'application. À ce stade, l’internaute peut se connecter à son compte \textbf{Ri3aya} ou créer un compte s’il n’en a pas. Lorsque l’internaute clique sur \textbf{Se connecter}, s'apparait l'écran indiqué à la deuxième figure pour fournir les informations nécessaires relatives au compte en question avec l'option \textbf{Mot de passe oublié}. Dans le cas où l'internaute n'a pas de compte, il doit cliquer sur \textbf{S'inscrire} de la première figure pour accéder aux écrans de création de compte indiqués aux deux dernières figures dans lesquels il doit fournir aussi des informations relatives au compte.
\begin{figure}[h]
	\includegraphics[scale=0.15]{D:/NTFS 3/Mon_master/M2/S4/Caputures_d_ecran_de_l_application/Screenshot_20210907-102900_HomeDoct}
	\includegraphics[scale=0.15]{D:/NTFS 3/Mon_master/M2/S4/Caputures_d_ecran_de_l_application/Screenshot_20210907-103417_HomeDoct}
	\includegraphics[scale=0.15]{D:/NTFS 3/Mon_master/M2/S4/Caputures_d_ecran_de_l_application/Screenshot_20210906-153840_HomeDoct}
	\includegraphics[scale=0.15]{D:/NTFS 3/Mon_master/M2/S4/Caputures_d_ecran_de_l_application/Screenshot_20210906-154413_HomeDoct}
	\centering
	\caption{Interfaces d'accueil pour l'internaute.}
	\label{Figure 5.2}
\end{figure}
\newline
En cas de création de compte, l’internaute doit fournie l’ensemble des informations requises, sinon, un message d’erreur lui sera affiché en dessous des champs vides ou dans lesquels les informations sont invalides syntaxiquement. Après avoir fournir ces informations, un code \gls{OTP} lui sera envoyé dans un SMS dont la détection est automatiquement faite via Firebase \footnote{Si la carte SIM est dans l'appareil de l'internaute, la détection du code \gls{OTP} sera en arrière-plan sans passer par l'écran indiqué à la première figure de \ref{Figure 5.3} mais plutôt il passe directement à la page suivante. Voir la figure}. Le code \gls{OTP} sert d’authentification d’internaute et de vérification de la validité du numéro de téléphone. Une fois le code \gls{OTP} validé, les informations du compte seront insérée dans la base de données.

\subsection{Espace Patient}
Lorsque l'internaute qui a sélectionnée "Patient " comme type de compte, dépasse la vérification du code \gls{OTP}, il passe à l'écran dans lequel il doit accomplir des informations relatives au type de compte chois. Voir la deuxième figure de \ref{Figure 5.3}.
\begin{figure}[h]
	\includegraphics[scale=0.2]{D:/NTFS 3/Mon_master/M2/S4/Caputures_d_ecran_de_l_application/Screenshot_20210906-155828_HomeDoct}
	\includegraphics[scale=0.2]{D:/NTFS 3/Mon_master/M2/S4/Caputures_d_ecran_de_l_application/Screenshot_20210906-155844_HomeDoct}
	\centering
	\caption{Interfaces d'accomplissement des informations de patient.}
	\label{Figure 5.3}
\end{figure}
\newline

\subsubsection{Options de navigation et paramétrage}
Après que le patient accomplis la création de son compte, il passe à la page d'accueil, dans laquelle il peut visualiser les consultants médicals disponibles, ainsi que ses rendez-vous en attente (en attendant les décisions des consultants médicals concernes), celles en cours et termines. Il peut aussi changer les paramétrés de son compte, visualiser l'historique de notifications,voir les informations du profil et les changer. Voir les figures \ref{Figure 5.4}
\begin{figure}[!h]
	\includegraphics[scale=0.2]{D:/NTFS 3/Mon_master/M2/S4/Caputures_d_ecran_de_l_application/Screenshot_20210906-155859_HomeDoct}
	\includegraphics[scale=0.2]{D:/NTFS 3/Mon_master/M2/S4/Caputures_d_ecran_de_l_application/Screenshot_20210906-160510_HomeDoct}
	\includegraphics[scale=0.2]{D:/NTFS 3/Mon_master/M2/S4/Caputures_d_ecran_de_l_application/Screenshot_20210906-160550_HomeDoct}
	\includegraphics[scale=0.2]{D:/NTFS 3/Mon_master/M2/S4/Caputures_d_ecran_de_l_application/Screenshot_20210906-155208_HomeDoct}
	\includegraphics[scale=0.2]{D:/NTFS 3/Mon_master/M2/S4/Caputures_d_ecran_de_l_application/Screenshot_20210906-155942_HomeDoct}
	\includegraphics[scale=0.2]{D:/NTFS 3/Mon_master/M2/S4/Caputures_d_ecran_de_l_application/Screenshot_20210906-155339_HomeDoct}
	\centering
	\caption{Interfaces des options de navigation et paramétrage.}
	\label{Figure 5.4}
\end{figure}

\subsubsection{Demande de rendez-vous}
Après avoir naviguer dans les consultations disponibles, le patient peut faire une demande de rendez-vous, et bien évident la demande peut être acceptée ou refusée comme déjà expliquer précédemment dans le diagramme d'activité \ref{Figure 4.5}. Voir les figures \ref{Figure 5.5}
\begin{figure}[!h]
	\includegraphics[scale=0.2]{D:/NTFS 3/Mon_master/M2/S4/Caputures_d_ecran_de_l_application/Screenshot_20210906-160023_HomeDoct}
	\includegraphics[scale=0.2]{D:/NTFS 3/Mon_master/M2/S4/Caputures_d_ecran_de_l_application/Screenshot_20210906-160031_HomeDoct}
	\includegraphics[scale=0.2]{D:/NTFS 3/Mon_master/M2/S4/Caputures_d_ecran_de_l_application/Screenshot_20210906-160709_HomeDoct}
	\includegraphics[scale=0.2]{D:/NTFS 3/Mon_master/M2/S4/Caputures_d_ecran_de_l_application/Screenshot_20210906-160447_HomeDoct}
	\includegraphics[scale=0.2]{D:/NTFS 3/Mon_master/M2/S4/Caputures_d_ecran_de_l_application/Screenshot_20210906-161549_HomeDoct}
	\centering
	\caption{Interfaces de demande de rendez-vous.}
	\label{Figure 5.5}
\end{figure}

\subsection{Espace Consultant Médical}
De même, si l'internaute qui a sélectionnée "Consultant Médical" comme type de compte, dépasse la vérification du code \gls{OTP}, il passe à l'écran dans lequel il doit accomplir des informations relatives au type de compte choisi. Elles comportent : la fonction, le lieu de travail, les horaires avec les dates, ainsi qu'une pièce de jointure confirmant l'identité professionnelle . Voir les figures \ref{Figure 5.6}.
\begin{figure}[!h]
	\includegraphics[scale=0.15]{D:/NTFS 3/Mon_master/M2/S4/Caputures_d_ecran_de_l_application/Screenshot_20210906-154534_HomeDoct}
	\includegraphics[scale=0.15]{D:/NTFS 3/Mon_master/M2/S4/Caputures_d_ecran_de_l_application/Screenshot_20210906-155011_HomeDoct}
	\includegraphics[scale=0.15]{D:/NTFS 3/Mon_master/M2/S4/Caputures_d_ecran_de_l_application/Screenshot_20210906-155038_HomeDoct}
	\includegraphics[scale=0.15]{D:/NTFS 3/Mon_master/M2/S4/Caputures_d_ecran_de_l_application/Screenshot_20210906-155059_HomeDoct}
	\centering
	\caption{Interfaces d'accomplissement des informations de consultant médical.}
	\label{Figure 5.6}
\end{figure}
\subsubsection{Options de navigation et paramétrage}
Après que le consultant médical accomplis la création de son compte, il passe à la page d'accueil, dans laquelle lui aussi peut visualiser les consultations en cours, celles en attentes, mais avec l'option de localiser ces patients ainsi que les rendez-vous termines. Il peut aussi changer les paramétrés de son compte, visualiser l'historique de notifications, voir les informations du profil et les changer exactement comme le patient. Voir les figures \ref{Figure 5.7}.
\begin{figure}[!h]
	\includegraphics[scale=0.15]{D:/NTFS 3/Mon_master/M2/S4/Caputures_d_ecran_de_l_application/Screenshot_20210908-135800_HomeDoct}
	\includegraphics[scale=0.15]{D:/NTFS 3/Mon_master/M2/S4/Caputures_d_ecran_de_l_application/Screenshot_20210906-161343_HomeDoct}
	\includegraphics[scale=0.15]{D:/NTFS 3/Mon_master/M2/S4/Caputures_d_ecran_de_l_application/Screenshot_20210906-161420_HomeDoct}
	\includegraphics[scale=0.15]{D:/NTFS 3/Mon_master/M2/S4/Caputures_d_ecran_de_l_application/Screenshot_20210906-161502_HomeDoct}
	\centering
	\caption{Interfaces des options de navigation et de paramétrage.}
	\label{Figure 5.7}
\end{figure}
\subsubsection{Approbation ou Refus d'une demande}
Le consultant médical est notifié de toute demande de rendez-vous, ensuite, il peut prendre sa décision concernant la consultation en question. Voir les deux premières figures de \ref{Figure 5.8}. En cas où le consultant médical accepte la demande de rendez-vous, un événement se créera dans le calendrier de son téléphone. Voir la dernière figure de \ref{Figure 5.8}
\begin{figure}[h]
	\includegraphics[scale=0.2]{D:/NTFS 3/Mon_master/M2/S4/Caputures_d_ecran_de_l_application/Screenshot_20210906-161317_HomeDoct}
	\includegraphics[scale=0.2]{D:/NTFS 3/Mon_master/M2/S4/Caputures_d_ecran_de_l_application/Screenshot_20210907-175157_HomeDoct}
	\includegraphics[scale=0.2]{D:/NTFS 3/Mon_master/M2/S4/Caputures_d_ecran_de_l_application/Screenshot_20210906-161448_Calendar}
	\centering
	\caption{Interfaces d'approbation et refus d'une demande de rendez-vous.}
	\label{Figure 5.8}
\end{figure}
\subsubsection{Mise à jour d'emploi du temps}
En cliquant sur l'icône "\textbf{+}" dans la page d'accueil du consultant médical, il lui apparait la calendrier du mois actuel pour sélectionner ses jours de travail, ensuite l'écran dans lequel il doit définie les horaires de travail. Cependant, il a la possibilité de voir les horaires sélectionnés au cas où il souhaite en supprimer ou en modifier un. 
\begin{figure}[h]
	\includegraphics[scale=0.2]{D:/NTFS 3/Mon_master/M2/S4/Caputures_d_ecran_de_l_application/Screenshot_20210906-163902_HomeDoct}
	\includegraphics[scale=0.2]{D:/NTFS 3/Mon_master/M2/S4/Caputures_d_ecran_de_l_application/Screenshot_20210906-163911_HomeDoct}
	\includegraphics[scale=0.2]{D:/NTFS 3/Mon_master/M2/S4/Caputures_d_ecran_de_l_application/Screenshot_20210906-163922_HomeDoct}
	\centering
	\caption{Interfaces de mise à jour d'emploi du temps.}
	\label{Figure 5.9}
\end{figure}
	
	
	\addtocontents{toc}{\protect\addvspace{.1em plus 1pt}}
	\bookmarksetup{startatroot}	
	
	\chapter{Conclusion et perspectives}
\label{chap:conclusion}

Pour conclure, l’objectif de mon projet de fin d’études consiste à concevoir et développer une application mobile permettant aux patients de solliciter des rendez-vous a domicile avec consultants médicals. Le tableau ci-dessous présente un bilan rétrospectif de ce que j'ai achevé des fonctionnalités définies dans la section \ref{4.1.1.1}.
\begin{table}[h]
	\begin{tabular}{|m{4cm}|m{10cm}|m{2cm}|}
		\hline
		\textbf{Espace} & \textbf{Fonctionnalités} & Achevée ? \\
		\hline
		Patient & -	Créer un compte en tant que Patient \newline
		- Réinitialiser le mot de passe \newline
		- Modifier les informations du profil \newline
		- Rechercher une consultation \newline
		- Demander un rendez-vous \newline
		- Payer les frais d’un rendez-vous \newline
		- Retirer une demande avant être accepter \newline
		- Visualiser la liste de ses rendez-vous \newline
		- Visualiser les détails d’un rendez-vous \newline
		- Visualiser le bilan statistique \newline
		- Visualiser les notifications reçues & 
		\begin{math} \color{green} \surd \end{math} \newline
		\begin{math} \color{green} \surd \end{math} \newline
		\begin{math} \color{green} \surd \end{math} \newline
		\begin{math} \color{green} \surd \end{math} \newline
		\begin{math} \color{green} \surd \end{math} \newline
		\begin{math} \color{red} \times \end{math} \newline
		\begin{math} \color{red} \times \end{math} \newline
		\begin{math} \color{green} \surd \end{math} \newline
		\begin{math} \color{green} \surd \end{math} \newline
		\begin{math} \color{green} \surd \end{math} \newline
		\begin{math} \color{green} \surd \end{math} \\
		\hline
		Consultant Médical & - Créer un compte en tant que Consultant Médical \newline
		- Réinitialiser le mot de passe \newline
		- Rectifier les informations du profil \newline
		- Visualiser la liste des rendez-vous \newline
		- Visualiser les informations d’un patient \newline
		- Accepter un rendez-vous \newline
		- Refuser un rendez-vous \newline
		- Planifier, initialement, les horaires de consultation \newline
		- Mettre à jour les horaires de consultations \newline
		- Reporter un rendez-vous pour un patient \newline
		- Visualiser le bilan statistique \newline
		- Visualiser les notifications reçues &
		\begin{math} \color{green} \surd \end{math} \newline
		\begin{math} \color{green} \surd \end{math} \newline
		\begin{math} \color{green} \surd \end{math} \newline
		\begin{math} \color{green} \surd \end{math} \newline
		\begin{math} \color{green} \surd \end{math} \newline
		\begin{math} \color{green} \surd \end{math} \newline
		\begin{math} \color{green} \surd \end{math} \newline
		\begin{math} \color{green} \surd \end{math} \newline
		\begin{math} \color{green} \surd \end{math} \newline
		\begin{math} \color{red} \times \end{math} \newline
		\begin{math} \color{green} \surd \end{math} \newline
		\begin{math} \color{green} \surd \end{math} \\
		\hline
	\end{tabular}
	\caption{Bilan rétrospectif sur les fonctionnalités implémentées.}
	\label{6.1}
\end{table}
D'après le tableau ci-dessus, j'ai implémenté 21 fonctionnalités sur 23, soit un pourcentage de 91\%, ce qui est tout à fait très significatif. \newline
En effet, ce stage et par conséquence, ce projet m'a été d’un apport primordial. Il m'a permis, entre autres, de découvrir beaucoup de concepts et d’outils que je n’ai jamais appris tout au long de mon cursus universitaire, notamment :
\begin{enumerate}
	\item Développement hybride : IONIC;
	\item Développement des web services C\#;
	\item Motifs d'architecture logicielle (\gls{MVC}, \gls{DAO} et \gls{SOA});
	\item Fonctions de hachage (SHA-256);
	\item Déploiement sur Azure;
	\item Test des services web (via Swagger \footnote{Swagger est un framework logiciel open source qui aide les développeurs à concevoir, construire, documenter et utiliser l’API Web RESTful. Pour en savoir plus, Pour en savoir plus, veuillez visiter les liens
	\href{https://docs.microsoft.com/en-us/aspnet/core/tutorials/web-api-help-pages-using-swagger?view=aspnetcore-5.0}{https://docs.microsoft.com/en-us/aspnet/core/tutorials/web-api-help-pages-using-swagger?view=aspnetcore-5.0} et 
	\href{https://www.nuget.org/packages/Swagger.Net.UI}{https://www.nuget.org/packages/Swagger.Net.UI}
	});
	\item La publication d’application Android sur Google Play Store;
	\item La sauvegarde et le partage de documents (Dropbox);
	\item L’environnement de rédaction LaTeX;
\end{enumerate}

Enfin, j'aimerai mentionner que l’objectif de \textbf{Ri3aya}, à long terme, est de développer toute une plateforme disponible sous forme d’application web, Android, iOS et pourquoi pas de bureau ? \newline
En effet, je crois, sincèrement, à ce projet que je trouve prometteur et je n’épargnerai aucun effort à son développement avec le temps.
	\chapter{Annexe 1 : Génération de l'App Bundle et l’APK en mode release}
\label{chap:annexe1}

Pour pouvoir générer l’\textbf{\gls{APK}} ou l'\textbf{App Bundle} en mode release, cliquez sur \textbf{Build} du menu horizontal d’Android Studio puis sur \textbf{Generate Signed Bundle/APK}. Une fenêtre s’ouvrira et vous serez demandé de choisir entre la génération d'App Bundle ou d’APK. Sélectionnez votre choix puis cliquez sur \textbf{Next}. Voir la première figure de \ref{Figure 7.1}. Dans la nouvelle fenêtre ouverte, cliquez sur \textbf{Create New} puis renseignez les informations demandées comme illustre la deuxième figure de \ref{Figure 7.1}.
\begin{figure}[!ht]
	\includegraphics[scale=0.6]{D:/NTFS 3/Mon_master/M2/S4/Caputures_d_ecran_de_l_application/4}
	\includegraphics[scale=0.6]{D:/NTFS 3/Mon_master/M2/S4/Caputures_d_ecran_de_l_application/Capture}
	\centering
	\caption{Choix de génération et génération d’un nouveau Key Store.}
	\label{Figure 7.1}
\end{figure}
\newline Une fois cliqué sur \textbf{Ok}, vous aurez l’interface \ref{Figure 7.2} qui présente un aperçu sur l’ensemble des informations précédemment saisies.
\begin{figure}[!ht]
	\includegraphics[scale=0.7]{D:/NTFS 3/Mon_master/M2/S4/Caputures_d_ecran_de_l_application/2}
	\centering
	\caption{Aperçu sur la configuration de l’APK ou de l'App Bundle.}
	\label{Figure 7.2}
\end{figure}
\newline Lorsque vous cliquez sur Next, vous aurez l’interface ci-dessous. Sélectionnez le mode release, puis cliquez sur Finish. La figure \ref{Figure 7.3} en présente une illustration.
\begin{figure}[!ht]
	\includegraphics[scale=0.7]{D:/NTFS 3/Mon_master/M2/S4/Caputures_d_ecran_de_l_application/3}
	\centering
	\caption{Options de génération de l'App Bundle ou de l’APK en mode release.}
	\label{Figure 7.3}
\end{figure}
\newline Le processus de génération de l'App Bundle ou de l’APK peut durer quelques minutes suivant la performance de votre ordinateur. Une fois achevé, vous serez notifiez à ce propos. L'App Bundle ou l’APK généré en mode release se trouve dans /android/app/release du dossier du projet.
	\chapter{Annexe 2 : Déploiement des Web API et de Base de données sur Azure}
\label{chap:annexe2}

Après avoir créer un compte Azure sur \href{https://portal.azure.com/}{https://portal.azure.com/}, commencez par créer une nouvelle base de données en cliquant sur \textbf{Base de données} dans le menu à gauche puis sur \textbf{Créer}, ce qui nécessite de sélectionnez un \textbf{Groupe de ressources} \footnote{Un groupe de ressources est un conteneur regroupant les applications web et les base de données} déjà existant ou en créer un nouveau. La création de la base de données nécessite également la création d'un nouveau serveur de base de données à partir duquel vous pouvez accéder à celle-ci. Enfin, réglez les paramètres de pare-feu et votre base de donnez sera prête. Les figures \ref{Figure 8.1} expliquent le processus.
\begin{figure}[!ht]
	\includegraphics[scale=0.17]{D:/NTFS 3/Mon_master/M2/S4/Azure/Capture}
	\includegraphics[scale=0.17]{D:/NTFS 3/Mon_master/M2/S4/Azure/Capture2}
	\includegraphics[scale=0.17]{D:/NTFS 3/Mon_master/M2/S4/Azure/Capture3}
	\includegraphics[scale=0.17]{D:/NTFS 3/Mon_master/M2/S4/Azure/Capture4}
	\includegraphics[scale=0.17]{D:/NTFS 3/Mon_master/M2/S4/Azure/Capture5}
	\centering
	\caption{Création d'une base de données sur Azure.}
	\label{Figure 8.1}
\end{figure}
\newline Juste après la creation de votre base de données, Azure vous générera le chaîne de connexion (Connection String) pour que votre projet ASP .NET \footnote{Le projet ASP .NET représente les web api} pouvoir connecter à celle-ci. Voir la figure \ref{Figure 8.2}.
\begin{figure}[!ht]
	\includegraphics[scale=0.37]{D:/NTFS 3/Mon_master/M2/S4/Azure/Capture6}
	\centering
	\caption{Génération du chaîne de connexion.}
	\label{Figure 8.2}
\end{figure}
\newline Pour déployer votre web api sur Azure, vous devez créer une nouvelle application web (Web App) en cliquant sur \textbf{App Services} dans le menu à gauche puis sur \textbf{Créer}. En suite sélectionnez le même groupe de ressources que celle créé lors de la création de votre base de données puis renseignez les informations demandées, une fois terminé cliquez sur \textbf{Vérifier + créer}. Enfin, sélectionnez le niveau de tarif, puis cliquez sur \textbf{Appliquer}. Voir les figures \ref{Figure 8.3}.
\begin{figure}[!ht]
	\includegraphics[scale=0.25]{D:/NTFS 3/Mon_master/M2/S4/Azure/Capture7}
	\includegraphics[scale=0.25]{D:/NTFS 3/Mon_master/M2/S4/Azure/Capture8}
	\includegraphics[scale=0.25]{D:/NTFS 3/Mon_master/M2/S4/Azure/Capture9}
	\centering
	\caption{Creation de l'application web.}
	\label{Figure 8.3}
\end{figure}
	\chapter{Annexe 3 : Publication d’application sur Google Play Store}
\label{chap:annexe3}

Après avoir créer un compte Google Play sur \href{https://play.google.com/console/}{https://play.google.com/console/}, cliquez sur \textbf{Create app} à droite de l'écran pour créer une nouvelle application puis renseignez les informations demandées et cliquez sur \textbf{Create app} en bas de l'écran. Ensuite, cliquez sur \textbf{Main store listing} dans le menu à gauche, puis renseignez les informations demandées et cliquez sur \textbf{Save}. Ensuite, cliquez sur \textbf{Production} dans le menu à gauche aussi, puis charger l'\gls{AAB} de votre application \footnote{Google ne n'accepte plus du format APK sur le Play Store. Pour en savoir plus, veillez visiter le lien \href{https://www.phonandroid.com/tout-savoir-sur-format-aab-android-app-bundles-google-play-store.html}{https://www.phonandroid.com/tout-savoir-sur-format-aab-android-app-bundles-google-play-store.html}}, puis renseignez les informations demandées. Une fois terminé, cliquez sur \textbf{Start rollout to Production}. Les figures \ref{Figure 9.1}, \ref{Figure 9.2} et \ref{Figure 9.3} expliquent toutes les étapes à suivre.
\begin{figure}[!ht]
	\includegraphics[scale=0.3]{D:/NTFS 3/Mon_master/M2/S4/Azure/screen}
	\includegraphics[scale=0.3]{D:/NTFS 3/Mon_master/M2/S4/Azure/Capture10}
	\centering
	\caption{Publication sur Google Play Store.}
	\label{Figure 9.1}
\end{figure}
\begin{figure}[!ht]
	\includegraphics[scale=0.25]{D:/NTFS 3/Mon_master/M2/S4/Azure/Capture14}
	\includegraphics[scale=0.25]{D:/NTFS 3/Mon_master/M2/S4/Azure/Capture15}
	\includegraphics[scale=0.25]{D:/NTFS 3/Mon_master/M2/S4/Azure/Capture16}
	\includegraphics[scale=0.25]{D:/NTFS 3/Mon_master/M2/S4/Azure/Capture17}
	\centering
	\caption{Publication sur Google Play Store 2.}
	\label{Figure 9.2}
\end{figure}
\begin{figure}[!ht]
	\includegraphics[scale=0.25]{D:/NTFS 3/Mon_master/M2/S4/Azure/Capture14}
	\includegraphics[scale=0.25]{D:/NTFS 3/Mon_master/M2/S4/Azure/Capture15}
	\includegraphics[scale=0.25]{D:/NTFS 3/Mon_master/M2/S4/Azure/Capture16}
	\includegraphics[scale=0.25]{D:/NTFS 3/Mon_master/M2/S4/Azure/Capture17}
	\centering
	\caption{Publication sur Google Play Store 3.}
	\label{Figure 9.3}
\end{figure}	
	
	
	\let\cleardoublepage\clearpage
	
	\chapter*{Bibliographie}
	\addcontentsline{toc}{chapter}{Bibliographie}
	\chaptermark{Références}
	\printbibliography[heading=none]
	
	\begin{enumerate}
		\item \bibliography{Véronique Messager Rota, Gestion de projet vers les méthodes agiles.}
		
		\item 	\bibliography{Rivest, Ronald L (1990), « The MD4 message digest algorithm », in : Conference on the Theory and
			Application of Cryptography, Springer, p. 303–311.}
		
		\item \bibliography{Rivest, Ronald et S Dusse (1992), The MD5 message-digest algorithm.}
		
		\item \bibliography{Anderson, Ross et Eli Biham (1996), « Tiger : A fast new hash function », in : International Workshop
			on Fast Software Encryption, Springer, p. 89–97.}
		
		\item 	\bibliography{Dobbertin, Hans, Antoon Bosselaers et Bart Preneel (1996), « RIPEMD-160 : A strengthened
			version of RIPEMD », in : International Workshop on Fast Software Encryption, Springer, p. 71–
			82.}
		
		\item \bibliography{Barreto, PSLM, Vincent Rijmen et al. (2000), « The Whirlpool hashing function », in : First open
			NESSIE Workshop, Leuven, Belgium, t. 13, p. 14.}
		
		\item \bibliography{Rivest, Ronald L et al. (2008), « The MD6 hash function–a proposal to NIST for SHA-3 », in :
			Submission to NIST 2.3, p. 1–234.}
		
		\item \bibliography{Aumasson, Jean-Philippe et al. (2008), « Sha-3 proposal blake », in : Submission to NIST 92.}
		
		\item \bibliography{Michail, Haralambos et al. (2005), « A low-power and high-throughput implementation of the SHA-1
			hash function », in : 2005 IEEE International Symposium on Circuits and Systems, IEEE, p. 4086–
			4089.}
		
		\item \bibliography{Glabb, Ryan et al. (2007), « Multi-mode operator for SHA-2 hash functions », in : journal of systems
			architecture 53.2-3, p. 127–138.}
		
		\item \bibliography{Dworkin, Morris J (2015), SHA-3 standard : Permutation-based hash and extendable-output functions,
			rapp. tech.}
		
		\item 	\bibliography{Dolmatov, Vasily et Alexey Degtyarev (2013), « GOST R 34.11-2012 : hash function », in : Independent
			Submission, Ed. Request for Comments : Updates 5831, p. 2070–1721.}
	\end{enumerate}
	
	

\end{document}