\chapter*{Conclusion et perspectives}
\addcontentsline{toc}{chapter}{Conclusion et perspectives}

\begin{comment}
	En d’autres termes, nous avons la version beta (test et amélioration avant sa publication) de
	l’application installée dans notre environnement de développement. Aussi, nous avons prévu la
	période pendant laquelle la solution finale sera déployée les plateformes de téléchargement.
	
	
	
	pour ce qui concerne les perspectives citons les points suivants :
	-réaliser les fonctionnalité non achevé;
	-abergement l'application sur google play sotre et app store;
	-ajouter plus des fonctionnalités.
\end{comment}
En guise de conclusion, nous pouvons affirmer que le développement mobile est un domaine enrichissant au niveau expérimental et qu'il est en évolution constante.
\newline

Ce stage a été l'occasion de faire le lien entre  connaissances académiques et le monde professionnel. Il nous a permis de développer nos compétences techniques, d'approfondir nos connaissaces théoriques et les mettre en pratique. Cette expérience a aiguisé nos capacités d'analyse et de synthése, et nous a permis de renforcer nos connaissances concernant le développement mobile.
\newline

Enfin, ce stage fut une expérience trés enrichissante sur les deux plans personnel et professionnel. En effet, il a été l'occasion de découvrir le dynamisme et l'enthousiasme qui caractérisent l'équipe de CADORIM. Les réunions réguliéres effectuées avec l'encadrant nous a permis à mettre en oeuvre les concepts de gestion de projets acquises. 
\newline
Nous avons implémenté 21 fonctionnalités sur 23, soit un pourcentage de $93 \%$ , ce qui est tout à fait très significatif.
\newline

 Pour ce qui concerne les perspectives citons les points suivants :
\begin{enumerate}
	\item[-]réaliser les fonctionnalité non achevé;
	\item[-]abergement l'application sur google play sotre et app store;
	\item[-]ajouter plus des fonctionnalités.
\end{enumerate}
