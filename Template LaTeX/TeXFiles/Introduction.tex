\let\cleardoublepage\clearpage

\chapter{Introduction générale}
\label{sec:DescriptionDuProjet}

%Dans cette section, je propose de voir dans un premier temps le contexte du sujet. Ensuite je ferais état de la problématique pour finir avec les objectifs à atteindre.

%\section{Contexte du projet}
La mise en relation entre les patients et les consultants médicaux consiste à collecter des informations sur eux, plus précisément leurs horaires disponibles pour faire des consultations aux maisons des patients. Comme ça nous offrons aux patients la possibilité de savoir les consultants médicaux disponibles à chaque instant.
\newline\newline Le fait de lier le patient et le consultant médical, a un double avantage, d'une part, pour le patient, car il va il peut savoir facilement des consultants médicaux disponibles, et d'autre part, pour le consultant, car il va obtenir un travail supplémentaire. Comme l'intitule du mémoire l'indique, notre Objectif est justement de lier les deux pôles de la problématique (Patient et Consultant médical).

\section{Motivations}

La demande de services médicaux devient de plus en plus importante en Mauritanie en tenant compte ceux qui nécessitent des visites à domicile. Les rendez-vous sont toujours organisés par des appels téléphoniques ou via les applications de messagerie, ce qui les rendent plus lents en terme d'organisation dû à l'existence d'un nombre peu important des fournisseurs connus de ces services ce qui n'est pas expliqué par la rareté de ces fournisseurs de façon général.

\section{Problématiques}

En réalité, la réalisation d’une application, fournissant les services sanitaires nécessaires, nécessite de faire face à des problématiques diverses et complexes. Ainsi, la société a décidé de se contenter, dans un premier temps, de la thématique de gestion des rendez-vous médicaux a domicile. \newline
Ce sujet soulève de nombreuses questions aux implications différentes. Tout d'abord que signifie un « Consultant médical » ?, le patient pourrait-il rechercher une consultant médical disponible ? Quelle procédure le patient doit-il suivre pour demander une consultation ? Comment pourrait-il payer les frais de consultation ? Comment pourrait-il reporter une consultation ?


\section{Objectifs}

La mise en relation entre les fournisseurs de services et les consommateurs (clients de façon générale) est un des services offerts par SMPNT. La société veut offrir un meilleur service aux clients à l'aide d'une application mobile qui gère la relation entre les patients et les consultants médicaux. Autrement dit, l'application permet d'organiser des rendez-vous à domicile.
