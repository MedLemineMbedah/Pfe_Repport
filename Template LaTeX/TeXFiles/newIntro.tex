
\chapter*{Introduction générale}
\addcontentsline{toc}{chapter}{Introduction générale}
Actuellement, à travers les progrès de la technologie, le smartphone est devenu un outil
indispensable de travail qui peut apporter un plus dans la vie professionnelle et sociale de tout
un chacun. D’ailleurs cela s’illustre parfaitement dans le domaine du Mobile Money qui est
l’une des plus grandes évolutions dans le secteur consacré aux échanges de capitaux notamment
en Afrique où une grande partie de la population n'a pas de compte bancaire.
Néanmoins, il existe encore beaucoup de progrès et d’innovation à effectuer dans ce secteur.
En effet, nous faisons face à un problème accru de manque de petites monnaies dans nos
échanges et transactions quotidiens entrainant des tensions entre acteurs et un ralentissement
des activités économiques particulièrement du secteur informel. Malgré l’évolution des outils
de transfert d’argent, cette question de petites monnaies fréquemment utilisées pour faciliter
nos petits achats reste encore un problème non résolu qui mérite une étude approfondie dans le
but d’en proposer une solution.