
\chapter*{Introduction générale}
\addcontentsline{toc}{chapter}{Introduction générale}
Actuellement, à travers les progrès de la technologie, le smartphone est devenu un outil
indispensable de travail qui peut apporter un plus dans la vie professionnelle et sociale de tout
un chacun. D’ailleurs cela s’illustre parfaitement dans le domaine du Mobile Money qui est
l’une des plus grandes évolutions dans le secteur consacré aux transfert d'argent 
de l’europe vers l'afrique où une grande partie de la population de 
l'afrique  n'a pas de compte bancaire.\newline
Au fur et à mesure on entend désormais partout, le processus KYC (Know Your Customer) se déploie dans de nombreux secteurs d’activité. Véritable atout pour combattre la fraude et l’usurpation d’identité, un processus KYC, lorsqu’il est correctement déployé, offre aujourd’hui une multitude d’avantages aux utilisateurs comme aux entreprises.\newline
Dans notre société, l’information est devenue un élément à la fois stratégique pour développer les
activités, et essentiel pour assurer un avantage concurrentiel (optimisation des coûts, meilleure
satisfaction client…) aux entités qui savent l’utiliser. C’est ce constat qui explique pourquoi les
entreprises cherchent aujourd’hui à mettre en place des systèmes de collecte et de traitement de
données toujours plus performants.
De même, la satisfaction du client est plus que jamais au centre des préoccupations des entreprises et
se concrétise par une gestion personnalisée de la relation client : comprendre les clients et leurs
attentes, les fdéliser, les inciter à consommer davantage. Le CRM, Customer Relationship
Management (GRC en français) a pour objet d'identifer, attirer et conserver les meilleurs clients et
d'en retirer chiffre d'affaire et rentabilité.
Ainsi le CRM englobe l'ensemble des activités et des processus que doit mettre en place une entreprise
pour interagir avec ses clients et ses prospects afn de leur fournir des produits et des services adéquats
au bon moment. Les entreprises ont de plus en plus recours à une approche de type de CRM, afn de se 
différencier. En effet, la banalisation de l'offre, une exigence accrue du client conduisent les
entreprises à faire évoluer leur offre dans le sens d'une plus grande personnalisation. Afn de parvenir
à cet objectif, l'entreprise est tenue de s'adapter à la profusion des canaux d'accès parallèles et en
particulier Internet.
L'arrivée des Nouvelles Technologies de l'Information et de la Communication a en effet un impact
très important sur les attitudes et les stratégies des entreprises face au CRM. Si bien que l'on peut se
demander si l'E-CRM, la gestion de la relation client par Internet constitue une véritable révolution
pour le CRM.





\begin{comment}
On l’entend désormais partout, le processus KYC (Know Your Customer) se déploie dans de nombreux secteurs d’activité. Véritable atout pour combattre la fraude et l’usurpation d’identité, un processus KYC, lorsqu’il est correctement déployé, offre aujourd’hui une multitude d’avantages aux utilisateurs comme aux entreprises. Mais quelles fonctions et quels usages se cachent derrière cet acronyme ? Comment a-t-il vu le jour et comment évolue-t-il face à la digitalisation des entreprises ? Découvrez, à travers ce guide, les atouts qu’un processus KYC automatisé peut apporter, ainsi que les différentes étapes qui en font une réussite.
\end{comment}