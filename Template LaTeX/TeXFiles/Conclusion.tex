\chapter{Conclusion et perspectives}
\label{chap:conclusion}

Pour conclure, l’objectif de mon projet de fin d’études consiste à concevoir et développer une application mobile permettant aux patients de solliciter des rendez-vous a domicile avec consultants médicals. Le tableau ci-dessous présente un bilan rétrospectif de ce que j'ai achevé des fonctionnalités définies dans la section \ref{4.1.1.1}.
\begin{table}[h]
	\begin{tabular}{|m{4cm}|m{10cm}|m{2cm}|}
		\hline
		\textbf{Espace} & \textbf{Fonctionnalités} & Achevée ? \\
		\hline
		Patient & -	Créer un compte en tant que Patient \newline
		- Réinitialiser le mot de passe \newline
		- Modifier les informations du profil \newline
		- Rechercher une consultation \newline
		- Demander un rendez-vous \newline
		- Payer les frais d’un rendez-vous \newline
		- Retirer une demande avant être accepter \newline
		- Visualiser la liste de ses rendez-vous \newline
		- Visualiser les détails d’un rendez-vous \newline
		- Visualiser le bilan statistique \newline
		- Visualiser les notifications reçues & 
		\begin{math} \color{green} \surd \end{math} \newline
		\begin{math} \color{green} \surd \end{math} \newline
		\begin{math} \color{green} \surd \end{math} \newline
		\begin{math} \color{green} \surd \end{math} \newline
		\begin{math} \color{green} \surd \end{math} \newline
		\begin{math} \color{red} \times \end{math} \newline
		\begin{math} \color{red} \times \end{math} \newline
		\begin{math} \color{green} \surd \end{math} \newline
		\begin{math} \color{green} \surd \end{math} \newline
		\begin{math} \color{green} \surd \end{math} \newline
		\begin{math} \color{green} \surd \end{math} \\
		\hline
		Consultant Médical & - Créer un compte en tant que Consultant Médical \newline
		- Réinitialiser le mot de passe \newline
		- Rectifier les informations du profil \newline
		- Visualiser la liste des rendez-vous \newline
		- Visualiser les informations d’un patient \newline
		- Accepter un rendez-vous \newline
		- Refuser un rendez-vous \newline
		- Planifier, initialement, les horaires de consultation \newline
		- Mettre à jour les horaires de consultations \newline
		- Reporter un rendez-vous pour un patient \newline
		- Visualiser le bilan statistique \newline
		- Visualiser les notifications reçues &
		\begin{math} \color{green} \surd \end{math} \newline
		\begin{math} \color{green} \surd \end{math} \newline
		\begin{math} \color{green} \surd \end{math} \newline
		\begin{math} \color{green} \surd \end{math} \newline
		\begin{math} \color{green} \surd \end{math} \newline
		\begin{math} \color{green} \surd \end{math} \newline
		\begin{math} \color{green} \surd \end{math} \newline
		\begin{math} \color{green} \surd \end{math} \newline
		\begin{math} \color{green} \surd \end{math} \newline
		\begin{math} \color{red} \times \end{math} \newline
		\begin{math} \color{green} \surd \end{math} \newline
		\begin{math} \color{green} \surd \end{math} \\
		\hline
	\end{tabular}
	\caption{Bilan rétrospectif sur les fonctionnalités implémentées.}
	\label{6.1}
\end{table}
D'après le tableau ci-dessus, j'ai implémenté 21 fonctionnalités sur 23, soit un pourcentage de 91\%, ce qui est tout à fait très significatif. \newline
En effet, ce stage et par conséquence, ce projet m'a été d’un apport primordial. Il m'a permis, entre autres, de découvrir beaucoup de concepts et d’outils que je n’ai jamais appris tout au long de mon cursus universitaire, notamment :
\begin{enumerate}
	\item Développement hybride : IONIC;
	\item Développement des web services C\#;
	\item Motifs d'architecture logicielle (\gls{MVC}, \gls{DAO} et \gls{SOA});
	\item Fonctions de hachage (SHA-256);
	\item Déploiement sur Azure;
	\item Test des services web (via Swagger \footnote{Swagger est un framework logiciel open source qui aide les développeurs à concevoir, construire, documenter et utiliser l’API Web RESTful. Pour en savoir plus, Pour en savoir plus, veuillez visiter les liens
	\href{https://docs.microsoft.com/en-us/aspnet/core/tutorials/web-api-help-pages-using-swagger?view=aspnetcore-5.0}{https://docs.microsoft.com/en-us/aspnet/core/tutorials/web-api-help-pages-using-swagger?view=aspnetcore-5.0} et 
	\href{https://www.nuget.org/packages/Swagger.Net.UI}{https://www.nuget.org/packages/Swagger.Net.UI}
	});
	\item La publication d’application Android sur Google Play Store;
	\item La sauvegarde et le partage de documents (Dropbox);
	\item L’environnement de rédaction LaTeX;
\end{enumerate}

Enfin, j'aimerai mentionner que l’objectif de \textbf{Ri3aya}, à long terme, est de développer toute une plateforme disponible sous forme d’application web, Android, iOS et pourquoi pas de bureau ? \newline
En effet, je crois, sincèrement, à ce projet que je trouve prometteur et je n’épargnerai aucun effort à son développement avec le temps.